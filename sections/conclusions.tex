\section{Συμπεράσματα}

Η εφαρμογή η οποία αναπτύχθηκε δίνει την δυνατότητα στον χρήστη: 

\begin{itemize}
	\item Να εξάγει τον χάρτη ενός χώρου και να τον χρησιμοποιεί.
	\item Να καλύπτει τον χάρτη με έναν ή δύο πράκτορες.
	\item Η μορφή του αισθητήρα μπορεί να πάρει οποιαδήποτε μορφή. Κώνος για κάμερα, κύκλος για αισθητήρα διοξειδίου του άνθρακα.
	\item Να διαλέγει στόχους αυτόνομα με 2 διαφορετικούς τρόπους. % Και οι εξαγόμενοι στόχοι να αξιολογούνται από συναρτήσεις κόστους προσαρμοζόμενων βαρών.
\end{itemize} 

\vspace{0.5 cm}

Η αξιολόγηση των παραπάνω δυνατοτήτων έγινε στο κεφάλαιο των πειραμάτων και κρίθηκε ικανοποιητική. Οι χάρτες που βγαίναν παρά κάποιες ατέλειες στάθηκαν αρκετοί για την περάτωση των πειραμάτων. Οι πράκτορες μπορούσαν επιτυχώς να εκτιμήσουν την θέση τους στον χώρο όπως συζητήθηκε σε προηγούμενα κεφάλαια.
Μάλιστα οι πράκτορες μπορούσαν πάντα να προσεγγίζουν τους ανατεθειμένους στόχους και δεν παρατηρήθηκε το φαινόμενο μη-προσέγγισης στόχων. Παρ' όλα αυτά σε συνθήκες εκτός προσομοίωσης αυτό είναι πολύ πιθανό. Οι planner που χρησιμοποιήθηκαν μπορούσαν να δίνουν διαδρομές αρκετά μακριά από τα εμπόδια.
Οι στόχοι που προέκυπταν όντως αξιολογούνταν και πάρα πολλές φορές ο στόχος που διάλεγαν οι πράκτορες ήταν όντως ο καλύτερος δυνατός. 


