\section{Παρουσίαση εργαλείων}

Σε αυτό το κεφάλαιο παρουσιάζονται κάποια εργαλεία διαθέσιμα για την διεκπεραίωση της διπλωματικής εργασίας. Τα εργαλεία αυτά είτε χρησιμοποιήθηκαν όντως στην διπλωματική με ή χωρίς τροποποιήσεις είτε απλά παρουσιάζονται χάριν πληρότητας. Θα γίνει ένας διαχωρισμός σε Yλικά (Hardware) και σε Yπολογιστικά (Software). 

%\subsection{Ultrabook Acer Swift 3 SF314-52}

%\paragraph{}Αυτός είναι ο υπολογιστής στον οποίο έγινε η πλειονότητα των πειραμάτων καθώς και ο μεγαλύτερο μέρος της συγγραφής του κώδικα. Διαθέτει Intel Core i7 επεξεργαστή RAM 8 GB, 256 GB SSD δίσκο, καθώς και κάρτα γραφικών Intel UHD Graphics 620. Link: \href{https://www.acer.com/ac/en/US/content/series/swift3}{\color{blue}{Acer Swift 3}}

%\begin{figure}[!ht]
%	\centering
%	\includegraphics[scale=1]{assets/images/Acer-Swift-3.png}
%	\caption{Ultrabook Used}
%	\label{fig: of something}
%	\end{figure}

\subsection{Turtlebot 2}

\paragraph{} Το Turtlebot είναι ένας χαμηλού κόστους ρομποτικός πράκτορας σχεδιασμένος για προσω-πική χρήση ό οποίος χρησιμοποιεί ανοιχτό κώδικα. Σχεδιάστηκε το 2010 από την Willow Garage. Διαθέτει μια κινητή βάση, την Kobuki base, καθώς και επιπλέον χώρο για επεκτάσεις ώστε να τοποθετούνται άλλα υλικά όπως φορητοί υπολογιστές, κάμερες, βραχ-ίωνες. Επίσης χάριν αυτονομίας παρέχεται και μπαταρία 2200 mAh. Παρόλα αυτά γίνεται να χρησιμοποιηθεί προφανώς και συνδεδεμένο στο ρεύμα. Στο παρακάτω σχήμα φαίνεται το Turtlebot 2. (Εικόνα \ref{fig:figure of something}). Link: \href{https://www.turtlebot.com/}{\color{blue}{Turtlebot 2}}

\vspace{1.5 cm}
\begin{figure}[!ht]
	\centering
	\includegraphics[scale=0.65]{assets/images/Turtlebot-2.png}
	\caption{Turtlebot-2}
	\label{fig:figure of something}
\end{figure}
 
 \newpage
 
\subsection{ROS - Robotics Operating System}
 
\paragraph{} Το ROS είναι ένα σύστημα ανοιχτού κώδικα σχεδιασμένο για εφαρμογές στην ρομποτική. Αν και δεν είναι ακριβώς λειτουργικό σύστημα (συνήθως αποκαλείται μεταλειτουργικό σύστημα - middleware), παρέχει υπηρεσίες που θα περίμενε κανείς από ένα λειτουργικό σύστημα όπως αφαιρετικότητα υλικού (Hardware Abstraction Level - HAL), χαμηλού επιπέ-δου έλεγχο, υλοποίηση συχνά \hspace{0.1 cm}χρησιμοποιούμενης\hspace{0.1 cm}λειτουργικότητας, ανταλλαγή μηνυμάτων μεταξύ των διεργασιών, καθώς και διαχείριση πακέτων. Επίσης παρέχει βιβλιοθήκες για λήψη, συγγραφή, και εκτέλεση κώδικα, σε διανεμημένα συστήματα. \\
\\
Οι στόχοι τού ROS είναι οι εξής: 
\begin{itemize}
 	\item Να είναι όσο πιο ελαφρύ γίνεται και ο γραμμένος κώδικας να χρησιμοποιείται και αλλού. Οι βιβλιοθήκες που παράγονται δηλαδή να γίνεται να χρησιμοποιηθούν και αλλού. 
 	\item  η παροχή και η βελτίωση των ROS client libraries όπως οι rospy, roscpp, roslisp ώστε ο καθένας να γράφει κώδικα όπου προτιμάει.
 	\item η παροχή πακέτων, χτισμένα σε μια ή σε συνδυασμό των παραπάνω client libraries, για την δημιουργία περισσότερων εφαρμογών.    
\end{itemize}
 
 Παρακάτω περιγράφονται τα βασικά δομικά στοιχεία του ROS \\
 
\subsubsection{Πακετα - Packages}

Τα πακέτα αποτελούν την κυριότερη δομή οργάνωσης υλικού στο ROS. Ένα πακέτο συνήθως περιέχει κώδικα, αρχεία τα οποία εκτελούν τον κώδικα τα οποία λέγονται launch files, configuration files, αρχεία τα οποία περιγράφουν την λειτουργικότητά του καθώς και εξαρτήσεις από άλλα πακέτα και τον τρόπο χτισίματος του πακέτου που είναι γραμμένος σε ένα αρχείο CMakeList. Το πιο απλό που μπορεί να κάνει κάποιος στο ROS είναι να φτιάξει και να απελευθερώσει (release) ένα πακέτο. To ROS παρέχει κάποια έτοιμα πακέτα για εκτίμηση θέσης, εξαγωγή ενός χάρτη που μπορούν να χρησιμοποιηθούν από οποιονδήποτε.
 
\subsubsection{Κόμβοι - Nodes} 

Ένας κόμβος είναι μια διαδικασία η οποία εκτελεί έναν υπολογισμό. Οι κόμβοι επικοινωνούν μεταξύ τους μέσω των υποδομών επικοινωνίας που περιγράφονται παρακάτω. Το ROS ως διανεμημένο σύστημα πολλές φορές διαχειρίζεται πολλούς κόμβους. Για παράδειγμα ένας κόμβος μπορεί να ελέγχει κάποιο laser, άλλος να χειρίζεται την κίνηση κάποιων τροχών, άλλος να συνδυάζει κάποια δεδομένα, άλλος να κάνει localization, και ούτω κάθε εξής. Ένας κόμβος γράφεται σε μια από τις παραπάνω ROS client libraries	και προκειμένου να συντονιστούν όλοι οι κόμβοι επικοινωνούν με τον ROS Master κόμβος ο οποίος αναφέρεται αμέσως μετά.  
 
\subsubsection{Υποδομές Επικοινωνίας} 

\paragraph{ROS Master κόμβος}
Ο ROS Master κόμβος παρέχει υπηρεσίες ονοματοδοσίας και εγγρα-φής στους υπόλοιπους κόμβους του συστήματος ROS. Καταγράφει και ελέγχει τους εκδότες (publishers) και τους παραλήπτες (subscribers) μηνυμάτων στα θέματα (topics), καθώς και στις υπηρεσίες. Ο ρόλος του Master είναι να δίνει την δυνατότητα στους υπόλοιπους κόμβους του συστήματος να βρίσκουν ο ένας τον άλλον. Από την στιγμή που οι κόμβοι εντοπιστούν μεταξύ τους, επικοινωνούν χωρίς την μεσολάβηση του ROS Master κόμβου. Επίσης ο ROS Master κόμβος παρέχει και τον Server παραμέτρων. Συχνά εκτελείται με την εντολή roscore αλλά στην περίπτωση παρουσίας launchfile η εντολή roscore δεν χρειάζεται.

\paragraph{Μετάδοση μηνυμάτων} 
Ένα σύστημα επικοινωνίας είναι από τα πρώτα ζητήματα που προκύπτουν όταν κάποιος\hspace{0.07 cm} θέλει να κατασκευάσει μια ρομποτική εφαρμογή. Το ROS παρέχει το πολλάκις\hspace{0.07 cm} ελεγμένο publish/subscribe\hspace{0.07 cm} (κοινοποίηση/εγγραφή) σύστημα. Δηλαδή μηνύμα-τα (messages) με πληροφορία, ανταλλάσσονται μεταξύ των κόμβων του συστήματος, τα οποία εκδίδονται σε ένα topic (θέμα) από έναν κόμβο και όποιος ενδιαφέρεται για τα εν λόγω μηνύματα, κάνει subscribe και τα διαβάζει. \\   

\paragraph{Ηχογράφηση μηνυμάτων} 
Επειδή το παραπάνω σύστημα είναι ασύγχρονο, τα μηνύματα μπορούν πολύ εύκολα να αποθηκευτούν και να αναπαραχθούν σε μια μελλοντική στιγμή χωρίς καμιά αλλαγή του κώδικα. 
Mε αυτό το σύστημα ο κάθε κόμβος αγνοεί πλήρως την προέλευση των μηνυμάτων. Αυτή η λειτουργικότητα παρέχεται μέσω του εργαλείου rosbag. \\

\paragraph{Σύγχρονη κλήση διαδικασιών} 
Πολλές φορές θέλουμε μια σύγχρονη απάντηση ενός κόμ-βου ή να εκτελεστεί ένας υπολογισμός επί τόπου. Σε αυτή την περίπτωση χρησιμοποιούμε την λειτουργικότητα υπηρεσία (service) οι οποία ορίζεται με παρόμοιο τρόπο με αυτή των μηνυμάτων. \\

\paragraph{Κατανεμημένο σύστημα παραμέτρων} 
Το ROS παρέχει την υπηρεσία αποθήκευσης κάποι-ων συχνά χρησιμοποιούμενων παραμέτρων στον λεγόμενο Parameter Server. Αυτές οι παράμετροι, οι οποίες συχνά είναι αρκετές στον αριθμό, είναι ορατές από κάθε κόμβο και μπορούν να χρησιμοποιηθούν μέσω των client libraries και εντός του κώδικα. \\


\subsubsection{Ειδικά χαρακτηριστικά για ρομπότ}

\paragraph{Μηνύματα αποκλειστικά για ρομπότ}
	Μετά από χρόνια πειραματισμού τελικά δημιουργή-θηκε η ανάγκη για την ύπαρξη ενός 
συνόλου μηνυμάτων που καλύπτουν τις συχνά εμφανιζό-μενες περιπτώσεις στην ρομποτική. Υπάρχουν ορισμοί μηνυμάτων που περιγράφουν την θέση του ρομπότ στον χώρο, άλλα για μετασχηματισμούς και διαφορετικά για διανύσματα. Επίσης υπάρχουν μηνύματα που περιγράφουν δεδομένα αισθητήρων όπως LIDAR, κάμερες και άλλα υλικά. Ακόμη παρέχονται μηνύματα πλοήγησης τα οποία εκφράζουν μονοπάτια χάρτες και κ.ο.κ. Η χρήση αυτών των μηνυμάτων βοηθάει στην εναρμόνιση της κάθε εφαρμογής στο ROS ecosystem. \\


\paragraph{Tf library} 
Το εν λόγω πακέτο ελέγχει και καταγράφει το που βρίσκεται κάποιο τμήμα (frame) του ρομπότ σε σχέση με κάποιο άλλο. Για παράδειγμα αν κάποιος θέλει να συνδυάσει δεδομένα κάμερας με δεδομένα laser, για να ξέρουμε που είναι ο κάθε αισθητήρας πρέπει να πάρει ένα κοινό καρέ (frame) αναφοράς. 
Η βιβλιοθήκη είναι πολύ βελτιστοποιημένη και διαχειρίζεται ρομπότ με δεκάδες βαθμούς ελευθερίας και ενημερώνει με συχνότητες των εκατοντάδων Hertz. Η tf library επιτρέπει μετασχηματισμούς στατικούς (πχ μια κάμερα στερεωμένη σε μια βάση) αλλά και δυναμικούς όπως στην περίπτωση των ρομποτικών βραχιόνων. Προφανώς οποιοσδήποτε κόμβος μπορεί να έχει πρόσβαση σε αυτή την πληροφορία.\\
Στην εικόνα Εικόνα \ref{fig:turtlebot tf} βλέπουμε πως είναι συνδεδεμένα διάφορα τμήματα ενός δένδρου για ένα σύστημα με ένα turtlebot που έχει στερεωμένο ένα LIDAR.
\begin{figure}[!ht]
	\centering
	\includegraphics[scale=0.2]{assets/images/tf-tree.png}
	\caption{Δένδρο μετασχηματισμού του Turtlebot}
	\label{fig:turtlebot tf}
\end{figure}

\paragraph{Γλώσσα περιγραφής Ρομπότ} 
	Αυτή η γλώσσα βοηθάει στην περιγραφή και μοντελοποίηση του πράκτορα σε ένα τρόπο 
κατανοητό από τον υπολογιστή και από τα υπόλοιπα εργαλεία του ROS. Ο τρόπος περιγραφής του ρομπότ είναι η URDF (Unified Robot Description Format), η οποία στην ουσία είναι ένα αρχείο XML το οποίο περιγράφει τις φυσικές ιδιότητες του ρομπότ, από μήκη εξαρτημάτων, μέγεθος τροχών, θέση αισθητήρων καθώς και την όψη του κάθε τμήματος του ρομπότ. \\

\paragraph{Κλήσεις Διαδικασιών με Δυνατότητα διακοπής}
Ενώ τα topics βοηθάν στην ασύγχρονη επικοινωνία, και τα services στην σύγχρονη. Πολλές φορές χρειάζεται να δώσουμε έναν στόχο, να παρακολουθήσουμε την πρόοδο του, να έχουμε την δυνατότητα να τον ακυρώσουμε αλλά και να ενημερωθούμε στην περίπτωση ολοκλήρωσής του.
Για αυτό τον λόγο το ROS παρέχει τις ενέργειες (actions). Έτσι κανείς μπορεί να δώσει έναν στόχο και στην πορεία να τον αλλάξει για οποιοδήποτε λόγο. Οι ενέργειες είναι μια πολύ ισχυρή ιδέα που χρησιμοποιείται σε όλο το οικοσύστημα του ROS.  \\

%%% NA grapsw edw meta gia actionlib????? %%%	 
\newpage
 
\subsubsection{Εργαλεία}

\paragraph{Εργαλεία γραμμής εντολών} 
Υπάρχουν πάνω από 45 εντολές για να δει κανείς την πορεία των εργασιών του, να ξεκινήσει κόμβους, να καταγράψει δεδομένα και πολλά άλλα πράματα. Σε περίπτωση που κανείς χρειάζεται γραφικό περιβάλλον παρέχονται τα παρακάτω γραφικά εργαλεία. \\

\begin{wrapfigure}{r}{0.5\textwidth}
	%\vspace{-15pt}
	%\setlength\figureheight{0.2\textwidth}
	%\setlength\figurewidth{0.4\textwidth}	
	\centering
	\includegraphics[scale=0.1]{assets/images/rviz_cropped.png}
	\caption{Γραφικό περιβάλλον rviz}
	\label{fig:rviz}
\end{wrapfigure}

%\begin{minipage}{0.5\textwidth}\raggedright
	\paragraph{rviz}
	Το πιο γνωστό από τα γραφικά εργαλεία είναι το rviz. (Εικόνα \ref{fig:rviz}) Οπτικοποιεί δεδομένα τριών διαστάσεων διαφόρων αισθητήρων, όπως κάμερες, laser, εικόνες, καθώς και παρουσιάζει οποιοδήποτε ρομπότ περιγράφεται με URDF. Επίσης δείχνει τα πάντα σε σχέση με ένα καρέ μετασχηματισμού (tf frame) της επιλογής μας μαζί με μια τρισδιάστατη όψη του ρομπότ μας.
%\end{minipage}

\paragraph{rqt} 
Το rqt παρέχει γραφικές διεπαφές για την παρακολούθηση του συστήματος. Προσφέρει το \textbf{rqt\_graph}, \textbf{rqt\_plot}, \textbf{rqt\_topic}, \textbf{rqt\_publisher}, \textbf{rqt\_bag} μεταξύ άλλων. \\
Το \textbf{rqt\_graph} μας δείχνει τους κόμβους και πως είναι διασυνδεδεμένοι μεταξύ τους, βοηθώντας πολύ στην αποσφαλμάτωση του συστήματος και στην βαθύτερη κατανόηση αυτού. \\
Το \textbf{rqt\_plot} παρουσιάζει αριθμητικά δεδομένα (πχ τάσης, κωδικοποιητών), τα οποία αλλάζουν συναρτήσει του χρόνου, σε μορφή γραφήματος. \\
Το \textbf{rqt\_topic} μας βοηθάει στο να παρακολουθήσουμε οποιοδήποτε αριθμό από topics και το \textbf{rqt\_publisher} μας επιτρέπει να στείλουμε τα δεδομένα σε κάποιο topic και να πειραματιστούμε με το σύστημα. \\
Το \textbf{rqt\_bag} καταγράφει τα δεδομένα ενός topic δίνοντας μας την δυνατότητα να τα επαναχρησιμοποιήσουμε.	
αυτούσια να εξάγουμε μια γραφική παράσταση ή ακόμα και να φτιάξουμε μια εικόνα. Στην Εικόνα \ref{fig:rqt_outputs} βλέπουμε κάποιες εξόδους των παραπάνω εργαλείων.
	
	
\begin{figure}[!hb]
	%\setlength\figureheight{0.2\textwidth}
	%\setlength\figurewidth{0.4\textwidth}	
	\centering
	\begin{subfigure}{0.49\textwidth}
		\includegraphics[scale=0.1]{assets/images/rqt-graph.png}
		\caption{Κόμβοι και πως συνδέονται μέσω του rqt graph}
	\end{subfigure}%
	\begin{subfigure}{0.49\textwidth}
		\includegraphics[scale=0.1]{assets/images/rqt-plot.png}
		\caption{Στιγμιότυπο από το rqt\_plot}
	\end{subfigure}
	\caption{Έξοδοι εργαλείων rqt}
	\label{fig:rqt_outputs}
\end{figure}	
	

\subsection{Βιβλιοθήκες Ρομποτικής}

Όπως αναφέρθηκε παραπάνω υπάρχουν ορισμένα θεμελιώδη προβλήματα τα στα οποία η ρομποτική προσπαθεί να προσφέρει όλο και καλύτερες λύσεις. Τα προβλήματα αυτά είναι η εξαγωγή χάρτη και ο προσανατολισμός του ρομπότ σε αυτόν (SLAM), η εύρεση της θέσης στον ήδη γνωστό χάρτη (localization), και η πλοήγηση του (navigation). Παρακάτω παρουσιάζονται οι βιβλιοθήκες που χρησιμοποιήθηκαν για την επίλυση αυτών των προβλη-μάτων.

\subsubsection{Gmapping}

Το Gmapping είναι ένα Rao-Blackwellized φίλτρο σωματιδίων (particle filter) που σκοπό έχει την κατασκευή (OGM) χρησιμοποιώντας τα δεδομένα κάποιου LIDAR. Τα φίλτρα σωματιδίων είναι μια βελτίωση των παλαιότερων και ελαφρότερων φίλτρων Kalman. Με την ανάπτυξη της ηλεκτρονικής και της μείωσης του μεγέθους των πλακετών πλέον μπορούμε να τρέξουμε πιο αποδοτικούς αλγορίθμους πάνω στα ρομπότ. Κάθε φίλτρο σωματιδίου έχει έναν χάρτη και μια θέση του ρομπότ σε αυτόν και ανά κάποιο χρονικό διάστημα τα σωματίδια συγκλίνουν και κρατιέται ο χάρτης με την μεγαλύτερη πιθανότητα. \\
Μία από τις προκλήσεις του αλγορίθμου είναι η εύρεση του ιδανικού αριθμού σωματιδίων ώστε και ο αλγόριθμος να εκτελείται γρήγορα αλλά και να εξάγει ένας ικανοποιητικός χάρτης. Το gmapping στον υπολογισμό κάθε επόμενου σωματιδίου λαμβάνει υπόψιν τους χάρτες των προηγούμενων με το μεγαλύτερο βάρος, έτσι κάθε επόμενο σωματίδιο έχει καλύτερη πληροφορία. Επίσης χρησιμοποιεί μια προσαρμοστική δειγματοληψία ώστε να απορρίπτει τα σωματίδια με το μικρότερο βάρος αποφεύγοντας έτσι το φαινόμενο του έλλει-ψης σωματιδίων (particle depletion).  

\begin{figure}[!h]
	\centering
	\includegraphics[scale=0.3]{assets/images/feb_map.png}
	\caption{Χάρτης Πλέγματος Πιθανοτήτων όπως προκύπτει από το Gmapping}
	\label{fig:gmapping map}
\end{figure}
 
 
\subsubsection{CRSM slam}

Αυτός ο αλγόριθμος ομοίως με τον Gmapping αντιστοιχεί σκαναρίσματα του laser πάνω στο χάρτη για να αποφύγει το αθροιστικό σφάλμα, μόνο που αντί για γενετικούς αλγορίθμους για την αντιστοίχιση χρησιμοποιεί τον Random Restart Hill Climbing (RRHC) \cite{Tsardoulias2013}. Στον τελικό υπολογισμό του χάρτη λαμβάνουν μέρος μόνο οι ακτίνες με την μεγαλύτερη χωρική πληροφορία και ο χάρτης ενημερώνεται δυναμικά ανάλογα με την μορφολογία του χώρου και την πυκνότητα των εμποδίων.  Επίσης ο RRHC παραμετροποιείται πολύ εύκολα για να μπορεί να χρησιμοποιηθεί ο CRSM σε πολλά διαφορετικά περιβάλλοντα. Όσον αναφορά το localization αξίζει να σημειωθεί πως ο εν λόγω αλγόριθμος δεν χρησιμοποιεί την οδομετρία ή την κινηματική του ρομπότ για τον υπολογισμό της θέσης του, μιας και η οδομετρία λόγω ολίσθησης τροχών εισάγει πολύ θόρυβο. Link: \href{http://wiki.ros.org/crsm_slam}{\color{blue}{CRSM slam}}
 
\subsubsection{Teb Local Planner}

Ο Teb Local Planner είναι μια προσθήκη του ήδη υπάρχοντα base local planner \cite{Gerkey}.
Η μέθοδος που επιστρατεύει - Ελαστικό Εύρος Χρόνου - βελτιστοποιεί τοπικά την πορεία του πράκτορα, ανάλογα με τον χρόνο εκτέλεσης, την απόσταση από εμπόδια, σε συμφωνία με τους κινητικούς περιορισμούς του πράκτορα \cite{Christoph2015}. Η παρούσα υλοποίηση δουλεύει και σε μή ολονομικά ρομπότ, δηλαδή για ρομπότ που μοιάζουν με αυτοκίνητα, ή διαφορικά ρομπότ. Η βέλτιστη διαδρομή υπολογίζεται επιλύνοντας ένα αραιό κλιμακούμενο πολλών σκοπών πρόβλημα βελτιστοποίησης. Ο χρήστης κατ' επιλογήν μπορεί να ρυθμίσει τα βάρη του αλγορίθμου βελτιστοποίησης στην περίπτωση συγκρουόμενων στόχων.
Link: \href{http://wiki.ros.org/teb_local_planner}{\color{blue}{Teb Local Planner}}

\subsubsection{AMCL}

Το AMCL είναι ένα σύστημα εντοπισμού θέσης για ένα ρομπότ που κινείται σε χώρο 2 διαστάσεων. Χρησιμοποιεί τον προσαρμοστικό (ή KLD - Sampling) Monte Carlo εντοπισμό στον χώρο, ο οποίος επιστρατεύει ένα φίλτρο σωματιδίων για να εντοπίσει την θέση και τον προσανατολισμό του ρομπότ πάνω σε έναν γνωστό χάρτη. Παρόμοια με gmapping όπου ένα σωματίδιο του φίλτρου είχε ένα χάρτη, εδώ ένα σωματίδιο έχει μία θέση και μια διεύθυνση του πράκτορα. Ο αλγόριθμος δουλεύει με δεδομένα laser. 
Παρακάτω (Εικόνα \ref{fig:amcl fig}),	 έχουμε μια οπτικοποίηση του αλγορίθμου στο προαναφερθέν rviz. Τα βελάκια μας δείχνουν την πιθανή θέση και κατεύθυνση του ρομπότ. Παρατηρούμε ότι πυκνώνουν (συγκλίνουν) όσο πλησιάζουμε την πραγματική τιμή. Παρόλα αυτά πάντα υπάρχει μια ασάφεια την οποία καλείται να αντιμετωπίσει οποιοσδήποτε αλγόριθμος χρησιμοποιεί την βιβλιοθήκη. Link: \href{http://wiki.ros.org/amcl}{\color{blue}{AMCL}}

\begin{figure}[!ht]
	\centering
	\includegraphics[scale=0.2]{assets/images/amcl-pose.png}
	\caption{Εκτίμηση Θέσης και κατεύθυνσης (με μια λέξη pose) ενός πράκτορα σε έναν γνωστό χάρτη}
	\label{fig:amcl fig}
\end{figure}
 
\subsubsection{ROS navigation stack}

Το navigation stack αποτελεί ίσως την πιο διαδεδομένη βιβλιοθήκη ρομποτικής. Σκοπό έχει την πλοήγηση ενός πράκτορα στον χώρο στέλνοντας τα κατάλληλα μηνύματα ταχύτητας στην κινητή βάση. Λέγεται stack γιατί στην ουσία αποτελεί ένα σύνολο πακέτων που αναλαμβάνουν να λύσουν τα προβλήματα της πλοήγησης. Άλλο πακέτο σχεδιάζει τα μονοπά-τια (dwa\_planner, navfn) άλλο πακέτο καθορίζει τις ασφαλείς θέσεις για το ρομπότ (cost-map\_2d), άλλο αποθηκεύει τους χάρτες (map\_server).
Παρακάτω περιγράφονται αναλυτικό-τερα τα εν λόγω πακέτα. 

\paragraph{costmap\_2d}

Αυτό το πακέτο παρέχει μια υλοποίηση ενός χάρτη κόστους (costmap) 2 διαστάσεων (ή και 3 διαστάσεων αν ενεργοποιηθεί το voxel grid) και λαμβάνοντας υπόψιν δεδομένα από laser και κάποιες εκ των προτέρων παραμετροποιήσεις δίνει κόστη στα κελιά. Ο χάρτης κόστους είναι δομή παρόμοια με τον Χάρτη Πλέγματος Πιθανοτήτων και αρκετές φορές έχουν το ίδιο μέγεθος. Επίσης αν ο χάρτης είναι γνωστός μπορεί να δοθεί μέσω του πακέτου map\_server και να φτιαχτεί αμέσως ο global costmap (επεξηγείται ευθύς παρακάτω). Τα κελιά παίρνουν τιμές μεταξύ 0 και 255 με το 0 να αναπαριστά τον ελεύθερο χώρο όπου ο πράκτορας μπορεί να διέλθει χωρίς πρόβλημα και 255 χώρος ο οποίος έχει οπωσδήποτε	εμπόδια και αν βρεθεί εκεί το ρομπότ βρίσκεται σίγουρα σε σύγκρουση. Κάποιες φορές δίνονται αρνητικές τιμές για άγνωστο χώρο.

\subparagraph{Είδη πλοήγησης} Υπάρχουν δύο είδη πλοήγησης η ολική (global) και η τοπική (local). Η ολική χρησιμοποιείται για να σχεδιαστεί ένα μονοπάτι μεταξύ του πράκτορα και στόχου σε μακρινή απόσταση και εκμεταλλεύεται το global costmap.
Αντίστοιχα η τοπική πλοήγηση που δημιουργεί μονοπάτια για αποτελεσματική αποφυγή εμποδίων χρησιμοποιεί το local costmap το οποίο κατασκευάζεται μόνο από τα δεδομένα του lidar. 

\subparagraph{Εισαγωγή και καθαρισμός εμποδίων} Ο χάρτης κόστους αυτόματα εγγράφεται στα θέματα (topics) αισθητήρων και ενημερώνεται αναλόγως. Ο κάθε αισθητήρας χρησιμοποιεί-ται είτε για να εισάγει ένα εμπόδιο στο χάρτη είτε για να αφαιρέσει είτε και τα δύο. Η εισαγωγή εμποδίου είναι απλά η αλλαγή της τιμής του αντίστοιχου κελιού στον χάρτη. H αφαίρεση εμποδίων όμως απαιτεί την μη ανίχνευση εμποδίων από κάποιον αισθητήρα. Αν χρησιμοποιείται μια 3 διαστάσεων δομή για την αποθήκευση της πληροφορίας εμποδίων, η πληροφορία προβάλλεται στις 2 διαστάσεις για την εισαγωγή στον χάρτη.

\subparagraph{Διάκριση χώρου} Αν και κάθε κελί στο χάρτη κόστους μπορεί να έχει μια από τις 256 διαφορετικές τιμές τελικά χρειάζεται να αναπαραστήσει μόλις τρεις καταστάσεις. Ελεύθερος χώρος, δεσμευμένος και άγνωστος. Κάθε κατάσταση έχει μια ειδική τιμή κόστους που της δίνεται με την προβολή της στον χάρτη. Υποστηρίζεται μόνο αναπαράσταση δύο διαστάσεων οπότε οι τιμές αποθηκεύονται σε στήλες. Στήλες με συγκεκριμένο αριθμό κατειλημμένων κελιών παίρνουν την τιμή \textit{ΕΠΙΚΙΝΔΥΝΟ\_ΕΜΠΟΔΙΟ (LETHAL\_OBSTACLE)}, στήλες με κάποιο αριθμό αγνώστων κελιών ανατίθεται η τιμή \textit{ΚΑΜΙΑ\_ΠΛΗΡΟΦΟΡΙΑ}, και σε όλα τα υπόλοιπα κελιά λαμβάνουν την τιμή \textit{ΕΛΕΥΘΕΡΟΣ\_ΧΩΡΟΣ}.

\subparagraph{Ενημέρωση χάρτη} Ο χάρτης ενημερώνεται με έναν ρυθμό που ορίζεται από την παρά-μετρο \textit{συχνότητα\_ενημέρωσης (update\_frequency)} που ρυθμίζεται από τον χρήστη. Κατά την διάρκεια ενός κύκλου γίνεται εισαγωγή και καθαρισμός εμποδίων όπως περιγράφτηκε παραπάνω και δίνονται οι κατάλληλες τιμές κόστους στις στήλες. Έπειτα συμβαίνει μια \textbf{προσαύξηση (inflation)} των τιμών των κελιών γύρω από τα κατειλημμένα κελιά με σκοπό ο πράκτορας να πηγαίνει όσο κοντά ή μακριά θέλουμε από τα εμπόδια. Η προσαύξηση γίνεται ανάλογα με την παράμετρο \textit{ακτίνα\_προσαύξησης (inflation\_radius)} και ορίζεται από τον χρήστη. 	

\subparagraph{Προσαύξηση (inflation)} Για το inflation που αναφέρθηκε πριν ορίζονται 5 συγκεκριμένα σύμβολα του χάρτη κόστους που αναφέρονται στο ρομπότ.  

\begin{itemize}
	\item "Lethal cost" σημαίνει πως υπάρχει όντως ένα εμπόδιο στο κελί και έτσι αν το κέντρο του ρομπότ ήταν όντως εκεί τότε θα ήταν σίγουρα σε σύγκρουση. 
	\item "Inscribed cost" σημαίνει ότι ένα κελί είναι πιο κοντά από την εσωτερική ακτίνα του ρομπότ. Έτσι το ρομπότ είναι σίγουρα σε σύγκρουση με κάποιο από τα γύρω εμπόδια αν βρεθεί σε κελί με τιμή ίση ή μεγαλύτερη του inscribed cost.
	\item "Possibly inscribed cost" είναι παρόμοιο με το inscribed cost με την διαφορά ότι ως ακτίνα αποκοπής λαμβάνεται η εξωτερική ακτίνα του ρομπότ. Έτσι αν το ρομπότ είναι σε κελί με τιμή ίση ή μεγαλύτερη του Possibly inscribed cost τότε το αν το ρομπότ βρεθεί σε σύγκρουση εξαρτάται από τον προσανατολισμό του ρομπότ. Ο όρος possibly (πιθανώς) χρησιμοποιείται γιατί μπορεί να μην είναι όντως ένα εμπόδιο αλλά κάποια προτίμηση του χρήστη, ο οποίος έθεσε αυτή την τιμή στον χάρτη. Για παράδειγμα αν ο χρήστης θέλει το ρομπότ να αποφύγει μια συγκεκριμένη περιοχή μπορεί να βάλει σε εκείνα τα κελιά τις δικές του τιμές στο χάρτη κόστους. Συνήθως η τιμή αυτή είναι το 128 αν και αυτό εξαρτάται από τα άνω και κάτω όρια των τιμών του χάρτη κόστους.
	\item "Unknown cell" σημαίνει πως το κελί είναι άγνωστο και ο χάρτης το ερμηνεύει όπως θέλει. 	
	\item "Free space" σημαίνει πως το κελί έχει τιμή 0 και δεν υπάρχει κάτι να εμποδίσει το ρομπότ να πάει εκεί.
\end{itemize} 

Παρακάτω βλέπουμε μια κάτοψη του τοπικού και ολικού χάρτη κόστους: Εικόνα \ref{fig:costmaps}.

\begin{figure}[!h]
	%\setlength\figureheight{0.2\textwidth}
	%\setlength\figurewidth{0.4\textwidth}	
	\centering
	\begin{subfigure}{0.49\textwidth}
		\includegraphics[scale=0.15]{assets/images/local-costmap.png}
		\caption{τοπικός χάρτης κόστους (local costmap)}
	\end{subfigure}%
	\begin{subfigure}{0.49\textwidth}
		\includegraphics[scale=0.15]{assets/images/global-costmap.png}
		\caption{Ολικός χάρτης κόστους (global costmap)}
	\end{subfigure}
	\caption{Costmaps}
	\label{fig:costmaps}
\end{figure}

\paragraph{move\_base}

Το πακέτο move\_base παρέχει την υλοποίηση ενός action (δες παράγραφο κλήση διαδικασιών με δυνατότητα διακοπής) το οποίο, δοσμένο ενός στόχου στον κόσμο, θα προσπαθήσει να μετατοπίσει την κινητή βάση προς τα εκεί. Ο κόμβος του move\_base ενώνει έναν τοπικό (local) και έναν ολικό (global) planner για να πετύχει τον στόχο της ολικής πλοήγησης. Επίσης διατηρεί δύο χάρτες κόστους έναν για το local planner και έναν για τον global planner σύμφωνα με την περιγραφή του παραπάνω πακέτου costmap\_2d. \\
\\
Το πακέτο παρέχει μια ROS διεπαφή για την ρύθμιση, εκτέλεση, και αλληλεπίδραση του ρομπότ με το navigation\_stack. Μια υψηλού επιπέδου απεικόνιση του move\_base φαίνεται παρακάτω (Εικόνα \ref{fig:move-base-nodes}). Οι μπλε κόμβοι αλλάζουν ανάλογα με την πλατφόρμα, οι γκρί είναι προαιρετικοί αλλά προσφέρονται για όλα τα συστήματα, και οι λευκοί απαιτούνται αλλά επίσης παρέχονται για όλα τα συστήματα. 

Εκτελώντας τον κόμβο του move\_base σε ένα ρομπότ που είναι σωστά ρυθμισμένο έχει ως αποτέλεσμα το ρομπότ να προσπαθεί να φτάσει έναν δοσμένο στόχο μέσα σε κάποια όρια ανοχής που έχει δώσει ο χρήστης ή να ειδοποιήσει τον χρήστη για την αποτυχία προσέγγισης του στόχου. Ο κόμβος κατ' επιλογήν εκτελεί προσπάθειες ανάκτησης αν θεωρήσει ότι το ρομπότ έχει κάπου κολλήσει. Από προεπιλογή, το move\_base θα εκτελέσει τις παρακάτω ενέργειες προκειμένου να μπορέσει να ελευθερώσει χώρο. \\

\begin{figure}[!h]
	\centering
	\includegraphics[scale=0.3]{assets/images/move-base-nodes.png}
	\caption{Οι κόμβοι του move\_base}
	\label{fig:move-base-nodes}
\end{figure}

Αρχικά, εμπόδια εκτός μιας ορισμένης από τον χρήστη περιοχή θα καθαριστούν από τον χάρτη του ρομπότ. Έπειτα το ρομπότ θα κάνει μια επιτόπια περιστροφή προκειμένου να ελευθερωθεί. Αν και αυτό αποτύχει, το ρομπότ θα καθαρίσει επιθετικά τον χάρτη, αφαιρώ-ντας όλα τα εμπόδια εκτός της τετράγωνης περιοχής στην οποία μπορεί να περιστραφεί. Τέλος ακολουθεί μια τελευταία επιτόπια περιστροφή. Αν όλα αποτύχουν ο στόχος θεωρείται απροσπέλαστος και αποστέλλεται μήνυμα αποτυχίας στον χρήστη. Παρακάτω φαίνεται το διάγραμμα ροής τον λεκτικά περιγραφόμενων ενεργειών: Εικόνα \ref{fig:recovery-behavour}.

\begin{figure}[!h]
	\centering
	\includegraphics[scale=0.5]{assets/images/recovery-behavour.png}
	\caption{Διαδικασία ανάκτησης πορείας}
	\label{fig:recovery-behavour}
\end{figure}


Στο επόμενο κεφάλαιο αναλύεται η υλοποίηση της εφαρμογής που αναπτύχθηκε

\newpage
