\section{Επισκόπηση Ερευνητική Περιοχής}

\subsection{Ρομποτική Εξερεύνηση}

	\paragraph{}Όπως γνωρίζουμε το πρόβλημα του Robot Exploration είναι
	 αρκετά παλιό και έχουν γίνει διάφορες προσπάθειες επίλυσης. 
	Χωρίζεται σε δύο κύρια υποπροβλήματα, το Mapping και το Localization (εύρεση θέσης). Ως Mapping ορίζεται η δυνατότητα χαρτογράφησης
	ενός πεπερασμένου χώρου (κενού ή με αντικείμενα εμπόδια εντός) από έναν ή περισσότερους αυτόνομους 
	πράκτορες (Robots). Ώς αυτόνομοι ορίζονται οι πράκτορες οι οποίοι δεν έχουν καμία ανθρώπινη επίβλεψη. 
	Όλη η διαδικασία γίνεται αυτόματα και μόνο μέσω προγραμμα-τισμού.  Το localization είναι το πρόβλημα της εκτίμησης 
	της θέσης του Robot σε σχέση με το περιβάλλον του και μάλιστα σύμφωνα με τον I.J. Cox\cite{Cox1991}, είναι ιδιαίτερα σημαντικό 
	και παίζει πρωταγωνιστικό ρόλο στην επιτυχή υλοποίηση εφαρμογών με έξυπνους κινούμε-νους αυτόνομους ρομποτικούς πράκτορες \\
	
	To 1997 o B. Yamauchi \cite{Yamauchi} όρισε την έννοια του frontier-based exploration μέσω της οποίας τα ρομπότ μπορούσαν από μόνα τους να κάνουν χαρτογράφηση. 
	Ως frontier ορίζεται μια περιοχή πάνω στο όριο μεταξύ του ελεύθερου χώρου (δηλαδή κάπου που μπορεί να πάει ο πράκτορας) και του ανεξερεύνητου χώρου. Μέχρι εκείνη την εποχή δεν υπήρχε τόσο έντονο ενδιαφέρον για το πρόβλημα του SLAM. Συνηθίζονταν τα ρομπότ να περιηγούνται σε ένα χώρο μόνο αν είχαν τον χάρτη γνωστό από πριν (να τον είχε φτιάξει ένας άνθρωπος για αυτά). Εντονότερο ενδιαφέρον για το SLAM προέκυψε από το 2000 και μετά. \\
	
	
%	Μέχρι εκείνη την εποχή τα ρομπότ μπορούσαν να περιηγηθούν σε ένα χώρο μόνο όμως αν είχαν τον χάρτη γνωστό 
%	από πριν (να τον είχε φτιάξει ένας άνθρωπος για αυτά). Η περιήγηση σε άγνωστα περιβάλλοντα ήταν ανέφικτη. \\
	
	Το 2001 ο Sebastian Thrun \cite{Thrun2001} με την ομάδα του συνέβαλλαν στον χώρο του localization. 
	Μέχρι το 2001 οι αλγόριθμοι προσπαθούσαν να επιλύσουν μόνο το πρόβλημα της εύρεσης θέσης με την χρήση φίλτρων Kalman 
	όμως αυτές οι προσεγγίσεις βασίζονταν σε κάποιες παραδοχές οι οποίες κάναν την πρακτική υλοποίηση δύσκολη. (Π.χ. ο θόρυβος στα φίλτρα Kalman 
	θεωρείται Gaussian). Οι προαναφερθέντες λοιπόν πρότειναν έναν πιθανοκρατικό αλγόριθμο localization τον Mixed Monte Carlo Localization,
	o οποίος προσπάθησε να δώσει λύση στο global localization και kidnapped robot (θεωρητική τηλεμεταφορά του ρομπότ σε μακρινό σημείο 
	και το ρομπότ πιστεύει ότι είναι στο αρχικό, πρακτικά σημαίνει προσπάθεια localization μετά από καταστροφική βλάβη 
	στον αρχικό αλγόριθμο localization) problem. Το Monte Carlo Localization παρουσιάζει τον κόσμο μέσω σωματιδίων που έχουν μια πιθανή λύση του προβλήματος. 
	Το Mixed Monte Carlo Localization βελτιώνει τον τρόπο με τον οποίο δημιουργούνται και αξιολογούνται τα εν λόγω σωματίδια. \\
	
	 Το 2005 οι Bugard, Stachnis \cite{Burgard2005} και οι συνεργάτες τους προσπάθησαν να εξερευνήσουν έναν άγνωστο χώρο με την χρήση πολλαπλών πρακτόρων. Όπως και στην περίπτωση του ενός πράκτορα, πάλι στόχος ήταν η ελαχιστοποίηση του χρόνου εξερεύνησης, απλά τώρα χρειαζόταν εξυπνότερη επιλογή στόχων και συντονισμός μεταξύ των πρακτόρων. Κάθε στόχος είχε ένα κόστος ξεχωριστό για τον κάθε πράκτορα. Τελικά τον στόχο προσέγγιζε ο πράκτορας του οποίου το κόστος ήταν πιο μικρό. Ταυτόχρονα έγινε προσπάθεια εφαρμογής της μεθόδου σε συνθήκες υποβέλτιστης επικοινωνίας μεταξύ των πρακτόρων. Κατά την διάρκεια της εξερεύνησης οι πράκτορες ένωναν τους διάφορους τοπικούς χάρτες που έφτια-χναν, μέχρις ώσπου να ολοκληρωθεί η εξερεύνηση. Στην παρούσα διπλωματική εφόσον ο αρχικός χάρτης είναι γνωστός, ενώση χαρτών χρησιμοποιήθηκε στο πλέγμα πεδίου κάλυψης όπου ένας κόμβος ενώνει σε ένα, τα δύο ξεχωριστά πεδία κάλυψης των δύο πρακτόρων που χρησιμοποιήθηκαν. \\
	
	
	Το 2008 οι Burgard, Stachnis, Wurm \cite{Wurm2008} πέραν του frontier based mapping χρησιμοποίη-σαν και την μέθοδο του segmentation, 
	δηλαδή χωρίσαν τον χάρτη σε υποτμήματα προς εξερεύνηση (π.χ. ξεχωριστά δωμάτια). Κάθε ρομπότ λοιπόν πάει και σε ένα ξεχωριστό segment 
	για να το εξερευνήσει. Το segmentation γίνεται με την χρήση διαγραμμάτων Voronoi. Αφου δημιουργηθεί το διάγραμμα, έπειτα χωρίζεται 
	σε τμήματα (segments), το οποία ανατίθενται στο κάθε ρομπότ προς εξερεύνηση. Η μέθοδος αυτή σύμφωνα με τους συγγραφείς οδήγησε σε 
	μικρότερο χρόνο περάτωσης του συνολικού χάρτη. Παρόμοιες μέθοδοι θα
	μπορούσαν να χρησιμοποιηθούν και σε προβλήματα κάλυψης. \\
	
	Μερικά χρόνια αργότερα το 2013 ο Μάνος Τσαρδούλιας \cite{2013},\cite{Tsardoulias2013} στη διδακτορική του διατριβή με θέμα την εύρεση θυμάτων σε αστικό περιβάλλον 
	προσβεβλημένο από φυσικές καταστροφές στην εξερεύνηση (με ένα όχημα) αναπαριστά τον χώρο μέσω ενός δικτύου πλέγματος πιθανοτήτων 
	συγκεκριμένα ενός OGM (Occupancy Grid Map). O πράκτορας έχει έναν αισθητήρα Lidar και κατα την ανανέωση του χάρτη συγκρίνεται το τρέχον σκανάρι-σμα 
	με τον ήδη υπάρχοντα χάρτη προς αποφυγίν αθροιστικών σφαλμάτων. To OGM (Occupancy Grid Map) έπειτα προσπελαύνεται μέσω του αλγορίθμου Brushfire. Όσο αναφορά 
	το localization, χρησιμοποείται η μέθοδος hill climbing προς επιτάχυνση του αλγορίθμου. Στην πλοήγηση χρησιοποιήθηκαν γενικευμένα 
	διαγράμματα Voronoi. Τέλος στην εξερεύνη-ση με πολλαπλούς πράκτορες χρησιμοποιήθηκε μια μέθοδος Master-Slave . 
	O Master έχει ID = 0 και είναι υπεύθυνος για την συγχώνευση των χαρτών και για την οργάνωση των ρομπότ slaves. \\
	
	Ένα χρόνο μετά το 2006 οι Andrew Howard , Gaurav S. Sukhatme του USC σε συνεργασί-α με την Lynne E. Parker του University of Tennessee \cite{Howard2006}, χρησιμοποίησαν 
	μια μεγάλη ομάδα από ετερογενή ρομπότ (80 στον αριθμό), στην προσπάθεια τους να χαρτογραφήσουν το εσωτερικό ενός κτηρίου και έπειτα να το 
	προστατέψουν από εισβολείς. Τα ρομπότ χωρίζονταν σε 2 ομάδες. Μια σχετικά μικρή ομάδα πολύ ικανών ρομπότ με δυνατότητες συντονισμού 
	της ομάδας και μια μεγαλύτερη ομάδα φθηνότερων μηχανημάτων εφοδιασμένα με κάμερα και μικρόφωνα ικανά για παρακολούθηση. Στο κομμάτι του 
	SLAM χρησιμοποι-ήθηκε και επίβλεψη από κεντρικό υπολογιστή, ο οποίος έκανε συνέχεια update τον χάρτη αλλά υπήρχε και αυτονομία, 
	δηλαδή ο κάθε πράκτορας έκανε τον δικό του χάρτη τοπικά με χρήση δικών του laser scans. Επίσης για να ομογενοποιηθούν τα δεδομένα, παρόμοια ή 
	κοντινά laser scans ενημερώνουν μια φορά τον global χάρτη. Το localization είναι πλήρως στην ευθύνη του κάθε robot (decentralized). 
	Το γεγονός αυτό έχει αρκετά πλεονεκτήματα (π.χ. ταχύτητα), όμως μειονεκτεί στο γεγονός ότι μπορεί το ίδιο robot να εξερευνήσει το 
	ίδιο μέρος 2 φορές.  \\
	
	%whatever stuff\cite{haney1954}
	
	\subsection{Ανάθεση Εργασιών}
	
	Ένα επίσης σημαντικό πρόβλημα προς επίλυση, είναι η κατανομή διάφορων άλλων εργασιών (tasks) που πρέπει να έρθουν εις πέρας, σε 
	ένα σενάριο συνεργασίας μεταξύ μίας ομάδας αυτόνομων ρομπότ, όπως είναι ο εντοπισμός και η μεταφορά διαφόρων αντικειμένων που βρίσκονται 
	μέσα στον χώρο. Το πρόβλημα αυτό είναι γνωστό ως Task Allocation και έχουν προταθεί διάφορες λύσεις για την επίλυσή του κατά τη 
	διάρκεια των χρόνων. \\
	
	Το 2000 στο \cite{Gerkey2000}, οι Brian P. Gerkey και Maja J Mataric πρότειναν μία υλοποίηση η οποία στηρίζεται στη 
	λογική publish/subscribe. Συγκεκριμένα, κάθε ρομπότ κάνει subscribe σε διάφορα subjects, τα οποία σχετίζονται με διάφορες 
	ικανότητες τις οποίες μπορεί να έχουν τα ρομπότ μίας ομάδας, όπως sonar, camera, speech, compass. Τα ρομπότ κάνουν subscribe 
	σε αυτά τα subjects, αν έχουν τις συγκεκριμένες ικανότητες. Στη συνέχεια, ένας χρήστης που χειρίζεται ένα κεντρικό σύστημα, 
	δημιουργεί μία λίστα με tasks. Ο αλγόριθμος ψάχνει μέσα στη λίστα με τα tasks και, ανάλογα με τις απαιτήσεις αυτών, 
	τα κάνει publish στα αντίστοιχα subjects. Τα ρομπότ που έχουν κάνει subscribe στα αντίστοιχα tasks, υπόλογίζουν ένα score, 
	που αντικατοπτρίζει την ικανότητα τους να φέρουν εις πέρας το task, με τη χρήση διάφορων μετρικών, και το κάνουν publish στο 
	αντίστοιχο subject. Στη συνέχεια, περιμένουν ώστε και τα υπόλοιπα ρομπότ να ολοκληρώσουν την ίδια διαδικασία. Το ρομπότ με 
	το μεγαλύτερο score είναι και αυτό που αναλαμβάνει το task. \\
	
	Το 2011 στο \cite{Eker2011}, οι Barıs Eker, Ergin Ozkucur, et al. πρότειναν τη 
	χρήση Partially Observable Markov Decision Processes (POMDP), για την μοντελοποίηση του προβλήματος της λήψης αποφάσεων από μία 
	ομάδα ρομπότ, που έχουν ως σκοπό την πλοήγηση σε ένα χώρο. Τα POMDPs είναι στην ουσία Markov Decision Processes (MDP), αλλά για περιπτώ-σεις 
	που δεν υπάρχουν πολλά δεδομένα για τον χώρο είτε όταν τα δεδομένα περιέχουν πολύ θόρυβο και δεν είναι αξιόπιστα. Επομένως, 
	τα POMDPs είναι κατάλληλα για την μοντελοποίηση προβλημάτων τα οποία περιέχουν αβεβαιότητα. Καθώς η λύση των Decentralized POMDPs είναι 
	αρκετά πολύπλοκη, στο paper προτείνεται η χρήση εξελικτικών στρατηγικών για την προσεγγιστική λύση του προβλήματος. Έτσι, μία ομάδα ρομπότ, 
	όσο διαφορετικά και αν είναι τα ρομπότ μεταξύ τους, μπορεί να μάθει μία συνολική, highlevel, στρατηγική, ώστε να αυξήσει το συνολικό reward 
	της ομάδας.\\
	
	 Το 2015 στο \cite{Wicke2015}, οι Drew Wicke, David Freelan και Sean Luke πρότειναν μία λύση βασισμένη στους bail bondsmen και bounty hunters.
	 Συγκεκριμένα, ένα από τα ρομπότ της ομάδας αναλαμβάνει το ρόλο του bail bondsman, ο οποίος είναι να ορίζει τις εργασίες προς διεκπεραίωση, τα υπόλοιπα ρομπότ το ρόλο του bounty hunter δηλαδή οι πράκτορες που προσπαθούν να <<εκτελέσουν την αποστολή>> και το προς ολοκλήρωση 
	 task το ρόλο του φυγά (fugitive). Ο bail bondsman, λοιπόν <<ποστάρει>> ένα task προς ολοκλήρωση, μαζί με το
	 bounty, δηλαδή την ανταμοιβή σε όποιο ρομπότ το ολοκληρώσει. Κάθε ρομπότ αποφασίζει αν θα κυνηγήσει την εργασία (task) ή όχι, ένα ή και
	 περισσότερα ρομπότ μπορούν να κυνηγούν ταυτόχρονα την ίδια εργασία, επομένως οι εργασίες δεν είναι πλέον αποκλει-στικές για κάποιο ρομπότ,
	 όπως γίνεται σε λύσεις βασισμένες σε δημοπρασία (auction based). Όσο μια εργασία δεν ολοκληρώνεται, τόσο ανεβαίνει το βραβείο της, επομένως γίνεται πιο ελκυστική
	 στα υπόλοιπα ρομπότ να το κυνηγήσουν και να την ολοκληρώσουν. Η συγκεκριμένη λύση αποδεικνύεται καλύτερη σε σχέση με τις στρατηγικές δημοπρασίας σε θορυβώδη και δυναμικά περιβάλλοντα, καθώς επίσης και σε καταστάσεις όπου δεν υπάρχει κάποια πληροφόρηση ή επικοινωνία 
	 μεταξύ των ρομπότ.\\
	 
	 Επίσης το 2015 στο \cite{luo2015}, οι Lingzhi Luo, Nilanjan Chakraborty και Katia Sycara προ-τείνουν μία λύση η οποία λαμβάνει υπόψιν το γεγονός
	 ότι κάθε ρομπότ μπορεί να αναλάβει ένα συγκεκριμένο αριθμό από tasks (λόγω περιορισμένης διάρκειας της μπαταρίας του) και κάποια tasks
	 μεταξύ τους μπορεί να μην είναι ανεξάρτητα, αλλά να σχηματίζουν ομάδες και να εξαρτώνται το ένα από το άλλο. Επίσης, για κάθε ρομπότ που
	 ολοκληρώνει ένα task, υπάρχει μία ανταμοιβή, και κάθε ρομπότ έχει ως σκοπό την μεγιστοποίηση της ανταμοιβής του, ολοκληρώνοντας όσα
	 περισσότερα tasks μπορεί, δεδομένου ότι έχει ένα άνω όριο tasks τα οποία μπορεί να αναλάβει, είτε γενικά, είτε μεταξύ της ίδιας ομάδας
	 tasks. Αποδεικνύεται τελικώς ότι, αν κάθε ρομπότ μεγιστοποιήσει την ατομική ανταμοιβή του, μεγιστοποιείται και η συνολική ανταμοιβή της
	 ομάδας των ρομπότ, δεδομένου ότι ένα ρομπότ μπορεί να αναλάβει ένα μόνο task. Προτείνεται μία centralized και μία distributed λύση, όπου
	 στην πρώτη, η ανταμοιβή για κάθε task βρίσκεται σε μία global, κοινή μνήμη και στη δεύτερη, κάθε ρομπότ έχει σε μία τοπική μεταβλητή την
	 ανταμοιβή για κάθε task και κάθε φορά που αναθέτει στον εαυτό του το task με την μεγαλύτερη ανταμοιβή, ενημερώνει και τα γειτονικά ρομπότ.
	 Η συγκεκριμένη λύση, είναι ενδεδειγμένη για ομάδες ετερογενών ρομπότ.\\

\subsection{Ρομποτική Κάλυψη}

	Η κάλυψη ενός χώρου, είναι ένα ζήτημα που μας απασχολεί όταν ο πράκτορας διαθέτει περισσότερους αισθητήρες από απλά ένα LIDAR, για να κάνει απλά χαρτογράφηση ή να αποφεύγει εμπόδια. Σε αυτή την περίπτωση χρησιμοποιείται συνήθως ένας Χάρτης Πλέγμα-τος Πιθανοτήτων (Occupancy Grid Map - OGM), όπου αναπαριστάται η πιθανότητα να έχουν καλύψει κάποιο ελεύθερο κελί οι υπόλοιποι αισθητήρες (κάμερα βάθους, αισθητήρας υπερήχων (sonar), κοκ). \\
	
	Το 2005 οι Xiaoming Zheng Sonal Jain \cite{Zheng2005} και οι ομάδα τους προσπάθησαν να καλύ-ψουν μια εκ των προτέρων γνωστή δασική περιοχή, με πολλαπλούς πράκτορες στον ελάχιστο χρόνο. Στην περίπτωση του ενός πράκτορα χρησιμοποιήθηκε ένα Χωρικό Δέντρο Κάλυψης (Spanning Tree Coverage - STC) και για πολλαπλούς πράκτορες το Πολλαπλών - πρακτόρων Χωρικό Δέντρο Κάλυψης (Multi-Robot Spanning Tree Coverage - MSTC). Και στις δύο περιπτώσεις ο χρόνος κάλυψης ήταν μακριά από τον βέλτιστο και οι συγγραφείς προτείναν την ευριστική Πολλαπλών-πρακτόρων Κάλυψη Δάσους (Multi-Robot Forest Coverage - MFC), ευρετική που πήγε πολύ καλύτερα από το όριο χειρότερης περίπτωσης. Ο χάρτης χωρίζοταν σε κελιά (σαν ΧΠΠ δηλαδή) και τα ρομπότ μπορούσαν να κινούνται είτε κάθετα είτε οριζόντια στον χάρτη. Σύμφωνα με τους συγγραφείς, συνδυασμός των MSTC και MFC μπορεί να δώσει ακόμα καλύτερα αποτελέσματα. \\
	
	Το 2009 το Εθνικό Πανεπιστήμιο της Νοτίου Κορέας \cite{Lee2009}, πρότεινε μια μέθοδο για τον καθαρισμό μεγάλων σε έκταση χώρων (αεροδρόμια, σταθμοί τρένων) από πολλούς πράκτορες. Στην προσέγγιση τους χώρισαν τον μεγάλο χώρο σε πολλούς μικρότερους τετράγωνους χώρους και ανάθεταν ένα χώρο σε κάθε ρομπότ. Οι χώροι ξεχωρίζονταν μεταξύ τους μέσω της ιδέας της Εικονικής Πόρτας (Virtual Door). Ως εικονική πόρτα ορίστηκε μιά ξαφνική αλλαγή στις διαστάσεις του χώρου (π.χ. ένας στενός σχετικά διάδρομος που ενώνει 2 χώρους). Στα ρομπότ ανατίθενται χώροι με βάση τον εκτιμώμενο χρόνο καθαρισμού και το καθένα ξεκινάει από την πάνω αριστερή γωνία του τετραγώνου και κάνει μαίανδρο μέχρι να καλυφθεί όλος ο ανατεθειμένος χώρος. Έπειτα ανατίθεται άλλος χώρος μέχρι να τελειώσει όλος ό μεγάλος αρχικός χώρος. \\
	
	Λίγα χρόνια μετά το 2014 πάλι στην Κορέα ο Tae-Shin Kim με τους συνεργάτες του \cite{Kim2014}, πρότειναν μια μέθοδο για ταυτόχρονη λήξη της κάλυψης από ετερογενείς πράκτορες σε μη-δομημένα περιβάλλοντα (unstructured environments). Το αρχικό περιβάλλον χωρίζεται σε χώρους προς επίσκεψη (LTV) και, λαμβάνοντας υπόψιν τις ταχύτητες των ρομπότ και των αποστάσεων του κάθε ρομπότ από τους χώρους προς επίσκεψη, ανατίθενται τελικά στόχοι. Η αποφυγή εμποδίων γίνεται μέσω μίας συνάρτησης απώθησης και η εύρεση μονοπατιού μέσω δέντρων Euler. \\

	Το 2017 ο Summit Gajjar et al. \cite{Gajjar2018} πρότειναν μια υλοποίηση για ολική κάλυψη ενός χώρου και βέλτιστο σχεδιασμό μονοπατιού (path planning) σε ένα εκ των προτέρων γνωστό περιβάλλον 2 διαστάσεων. Στόχος ήταν να περάσει ο πράκτορας απ' όλα τα σημεία του χώρου και ο χώρος να καλυφθεί στον ελάχιστο χρόνο. Ο αλγόριθμος που προτείνεται είναι ο Σχεδιασμός Μονοπατιού Ολικής Κάλυψης (Complete Coverage Path Planning - CCPP) και χρησιμοποιεί συναρτήσεις κόστους για να βρει το μονοπάτι που επισκέπτεται το κάθε σημείο μια φορά. Το κόστος κάθε κελιού είναι ο αριθμός των εμποδίων που το περιβάλλει. Ο αλγόριθμος δούλεψε για εμπόδια διαφορετικών διαστάσεων. \\ 	

	Τέλος το 2018, ο Suruz Miah και ο Jacob Knoll \cite{Miah2018}, προσπάθησαν αν βελτιστοποιήσουν τον χρόνο κάλυψης ενός 2D χώρου από μια ομάδα από ετερογενή ρομπότ ορίζοντας ένα μη ομοιόμορφο βαθμωτό μέγεθος: την πυκνότητα. Η αρχική θέση των πρακτόρων καθόριζε το πόσο γρήγορα θα συνέκλιναν οι πράκτορες στις τελικές θέσεις. Τελικά το πρόβλημα έπαιρνε την μορφή εύρεσης ολικού ελαχίστου το οποίο ήταν και το κέντρο μιας περιοχής Voronoi.
	
	Στο επόμενο κεφάλαιο αναλύονται τα εργαλεία που χρησιμοποιήθηκαν.
