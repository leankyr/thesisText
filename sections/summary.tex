\section*{Περίληψη}

Η επιστήμη της ρομποτικής είναι ένας κλάδος που τις τελευταίες δεκαετίες παρουσιάζει σημαντικό ενδιαφέρον. Το πρόβλημα της κάλυψης χώρων και της εύρεσης θέσης ενός πράκτορα μέσα σε αυτόν είναι από τα πρώτα που απασχόλησαν τους ερευνητές. Το εν λόγω πρόβλημα είναι εξίσου ενδιαφέρον σήμερα, και πέραν των τετριμμένων υλοποιήσεων ενός πράκτορα, προτείνονται και επιλύσεις πολλαπλών πρακτόρων. Αυτές οι λύσεις εφαρμόζο-νται σε διάφορα προβλήματα όπως παραδείγματος χάριν στρατιωτικές εφαρμογές, σε εφαρ-μογές εύρεσης και διάσωσης εντός αστικού περιβάλλοντος (USAR), καθώς και σε εφαρμογές παρακολούθησης χώρων. Αυτές οι προσπάθειες είναι ακόμα πιο προσιτές την τελευταία δεκαετία χάριν της ανάπτυξης της υπολογιστικής δύναμης, της πτώσης των τιμών του υλικού και της αύξησης του εύρους ζώνης των δικτύων.

Η παρούσα διπλωματική εργασία λοιπόν παρουσιάζει μια μέθοδο συντονισμού δύο ρομποτικών πρακτόρων με σκοπό την πλήρη κάλυψη ενός χώρου μέσω ενός αισθητήρα ενδιαφέροντος. Στο πρώτο κεφάλαιο γίνεται η εισαγωγή, όπου παρουσιάζεται ο σκοπός της διπλωματικής και η διάρθρωση του κειμένου. Στο δεύτερο κεφάλαιο γίνεται η επισκόπηση της ερευνητικής περιοχής και πιο συγκεκριμένα των χώρων της ρομποτικής εξερεύνησης, ανάθεσης εργασιών και της ρομποτικής κάλυψης. Στο τρίτο κεφάλαιο παρουσιάζονται τα εργαλεία που χρησιμοποιήθηκαν δηλαδή το υλικό (Hardware), το ROS, και οι βιβλιοθήκες που παρέχει. Στο τέταρτο κεφάλαιο αναλύεται η μέθοδος που χρησιμοποιήθηκε δηλαδή ο τοπολογικός γράφος καθώς και η προσέγγιση κοντινότερου σημείου για την επίλυση του προβλήματος. Στο πέμπτο κεφάλαιο παρουσιάζεται ο τρόπος διεξαγωγής των πειραμάτων και τέλος αναφέρονται οι δυσκολίες που παρουσιάστηκαν, προοπτικές για μελλοντική μελέ-τη και τα συμπεράσματα που προέκυψαν. 

Για την προσομοίωση χρησιμοποιήθηκε ο προσομοιωτής Gazebo και καθώς και το rviz. Για την συγγραφή του κώδικα χρησιμοποιήθηκε η γλώσσα Python. 


