\section{Διεξαγωγή πειραμάτων}

Σε αυτή την ενότητα παρουσιάζεται ο τρόπος διεξαγωγής πειραμάτων και η φιλοσοφία πίσω από αυτόν. Πειράματα έγιναν με χρήση ενός πράκτορα ή δύο πρακτόρων και σε δύο διαφορετικούς χάρτες, έναν μικρού μεγέθους και έναν μεγάλου μεγέθους. Τα αποτελέσματα θα παρουσιαστούν σε μορφή διαγραμμάτων καθώς επίσης και σε μορφή αριθμητικής πληρο-φορίας. Αρχικά οι χάρτες εξερευνήθηκαν με τυχαία επιλογή στόχων σαν βάση σύγκρισης χρόνων και έπειτα με την μέθοδο που προτείνουμε. Τέλος θα ακολουθήσει σχολιασμός πάνω σε αυτά. Αρχικά όμως ας παρουσιάσουμε τους δύο χάρτες. Και οι δύο χάρτες είναι τμήματα του κόσμου Willow Garage του προσομοιωτή Gazebo. Αρχικά ο μικρός, Εικόνα: \ref{fig:Small map} και έπειτα ο μεγάλος. Εικόνα: \ref{fig:Big map}.
Ο μικρός χάρτης έχει έκταση 289 $m^2$ και ο μεγάλος έχει έκταση  832 $ m^2 $.

\begin{figure}[!h]
	\centering
	\includegraphics[scale=0.3]{assets/images/WillowGarageWorldSmallMap.png}
	\caption{Μικρός χάρτης διεξαγωγής πειραμάτων}
	\label{fig:Small map}
\end{figure}

\begin{figure}[!h]
	\centering
	\includegraphics[scale=0.3]{assets/images/WillowGarageWorldBigMap.png}
	\caption{Μεγάλος χάρτης διεξαγωγής πειραμάτων}
	\label{fig:Big map}
\end{figure}

\newpage

\subsection{Ultrabook Acer Swift 3 SF314-52}

\paragraph{}Αυτός είναι ο υπολογιστής στον οποίο έγινε η πλειονότητα των πειραμάτων καθώς και το μεγαλύτερο μέρος της συγγραφής του κώδικα. Διαθέτει Intel Core i7 επεξεργαστή RAM 8 GB, 256 GB SSD δίσκο, καθώς και κάρτα γραφικών Intel UHD Graphics 620. \\
Στον παρακάτω σύνδεσμο περιέχονται περισσότερες πληροφορίες: \\
Link: \href{https://www.acer.com/ac/en/US/content/series/swift3}{\color{blue}{Acer Swift 3}}

%\begin{figure}[!ht]
%	\centering
%	\includegraphics[scale=1]{assets/images/Acer-Swift-3.png}
%	\caption{Ultrabook Used}
%	\label{fig:figure of PC}
%	\end{figure}





\subsection{Πειράματα ενός πράκτορα - Μικρός Χάρτης}

Αρχικά θα παρουσιάσουμε την τελική κατάσταση του μικρού χάρτη η οποία φαίνεται στην επόμενη εικόνα (Εικόνα 23). Το ροζ είναι το επίθεμα της κάλυψης

\begin{figure}[!h]
	\centering
	\includegraphics[scale=0.3]{assets/images/final-map-one-robot-teb-planner.png}
	\caption{Τελική μορφή μικρού χάρτη μετά την ολοκλήρωση της κάλυψης}
	\label{fig:target at cov limits}
\end{figure}

Παρατηρούμε ότι μένουν κάποια μικρά κενά σε γωνίες και σε σημεία κοντά σε τοίχο. Αυτό είναι φυσιολογικό και αναμενόμενο γιατί όπως είπαμε, εξαιτίας της μεθόδου \textit{filter\_goal} στόχοι που προκύπτουν κοντά σε τοίχο απορρίπτονται και o τοπολογικός γράφος από την φύση του βγάζει στόχους αρκετά μακριά από τοίχους. Επίσης η μέτρηση χρόνου και ποσοστού έχει σταματήσει λίγο πιο πριν μιας και η παραπάνω εικόνα είναι σε στάδιο πολύ πέρα από το $ 95\% $ κάλυψης. Τέλος φαίνονται κάποιοι προηγούμενοι στόχοι του γράφου καθώς επίσης και η μορφή του τοπικού χάρτη κόστους (local\_costmap) που κινείται μαζί με το robot. 

\newpage
\subsubsection{Τυχαία επιλογή στόχων}

%\begin{figure}[!h]	
%	\includegraphics[scale=0.3]{assets/figures/rand-one-med-map-1.png}
%	\caption{κάλυψη ανά χρόνο μικρός χάρτης 1 πράκτορας τυχαία πείραμα 1}
%\end{figure}

%\newpage

\begin{figure}[!h]
	\centering
	\includegraphics[scale=0.3]{assets/figures/rand-one-med-map-2.png}
	\caption{Κάλυψη ανά χρόνο, μικρός χάρτης, 1 πράκτορας, τυχαία, ενδεικτικό πείραμα}
\end{figure}

%\begin{figure}[!h]
%	\centering
%	\includegraphics[scale=0.3]{assets/figures/rand-one-med-map-3.png}
%	\caption{κάλυψη ανά χρόνο μικρός χάρτης 1 πράκτορας τυχαία πείραμα 3}
%\end{figure}

\begingroup
\centering
\begin{tabular}{c | c }
	\textbf{Λεκτική περιγραφή πειράματος} & \textbf{Χρόνοι}\\ \hline{}
	Καλύτερη επίδοση & 1365.899 sec \\ \hline
	Μέση επίδοση & 1561.642 sec   \\ \hline
	Χειρότερη επίδοση & 1693.71 sec \\
\end{tabular}
\captionof{table}{Μικρός χάρτης, 1 πράκτορας, τυχαία σύνοψη}
\label{tab:rand 1 small}
\endgroup
\paragraph{Σχολιασμός Πειραμάτων}


Όπως φαίνεται από την καμπύλη, η εξέλιξη της κάλυψης δεν είναι εντελώς ομαλή. Γνωρίζουμε \cite{2013},\cite{Tsardoulias2013} ότι αυτές οι καμπύλες έχουν μορφή λογαριθμικής καμπύλης. Αυτό τείνει να εμφανιστεί περισσότερο στο τρίτο πείραμα αλλά στα υπόλοιπα δύο είμαστε μακριά από αυτή την μορφή. Αυτό συμβαίνει επειδή λόγω της τυχαιότητας στο μέσον της διαδικασίας εκεί που αρχίζουν να δυσκολεύει η επιλογή στόχων βλέπουμε ότι ο αλγόριθμος αποτυγχάνει να επιλέξει καλούς στόχους με αποτέλεσμα ο πράκτορας να περνάει πολλές φορές από κεκαλυμμένο χώρο. Αυτό το φαινόμενο κάπως εξομαλύνεται στην περίπτωση της έξυπνης επιλογής στόχων. Τέλος παρατηρούμε ότι χρειαζόμαστε κάπου στα είκοσι έξι λεπτά για να καλύψουμε τον χώρο αυτόν.

%%%%%%%%%%%%%%%%%%%%%%%%%%%%%%%%%%%%%%%%%%%%%%%%%%%%%%%%%%%%%%%%%%%%%%%%%%%%%%
%%%%%%%%%%%%%%%%%%%% Four Gains one rob small map %%%%%%%%%%%%%%%%%%%%%%%%%%%%
%%%%%%%%%%%%%%%%%%%%%%%%%%%%%%%%%%%%%%%%%%%%%%%%%%%%%%%%%%%%%%%%%%%%%%%%%%%%%%

\subsubsection{Επιλογή στόχων μέσω κοντινότερου κόμβου}

Εδώ παρουσιάζεται το πείραμα όπου η επιλογή στόχων έγινε με την μέθοδο του κοντινότερου κόμβου. Θα δούμε ότι οι χρόνοι είναι σαφώς καλύτεροι σε σχέση με το προηγούμενο πείραμα. Πίνακας \ref{tab:rand 1 small}.


%\begin{figure}[!h]	
%	\includegraphics[scale=0.3]{assets/figures/four-one-med-map-1.png}
%	\caption{Κάλυψη ανά χρόνο μικρός χάρτης 1 πράκτορας κοντινότερος κόμβος πείραμα 1}
%\end{figure}

\begin{figure}[!h]
	\centering
	\includegraphics[scale=0.3]{assets/figures/four-one-med-map-3.png}
	\caption{Κάλυψη ανά χρόνο, μικρός χάρτης, 1 πράκτορας, κοντινότερος κόμβος, ενδεικτικό πείραμα}
\end{figure}

%\begin{figure}[!h]
%	\centering
%	\includegraphics[scale=0.3]{assets/figures/four-one-med-map-3.png}
%	\caption{κάλυψη ανά χρόνο μικρός χάρτης 1 πράκτορας κοντινότερος κόμβος πείραμα 3}
%\end{figure}


\newpage

\begingroup
\centering
\begin{tabular}{c | c }
	\textbf{Λεκτική περιγραφή πειράματος} & \textbf{Χρόνοι}\\ \hline{}
	Καλύτερη επίδοση & 790.553 sec \\ \hline
	Μέση επίδοση & 931.56 sec   \\ \hline
	Χειρότερη επίδοση & 1086.215 sec \\ 
\end{tabular}
\captionof{table}{Μικρός χάρτης 1 πράκτορας κοντινότερος κόμβος σύνοψη}
\endgroup

\paragraph{Σχολιασμός Πειραμάτων}



Εδώ παρατηρούμε σαφώς καλύτερους χρόνους σε σχέση με πριν μιας και τώρα η μέθοδος επιλογής στόχων είναι έξυπνη. Ακόμη βλέπουμε πιο απότομη αλλαγή της κάλυψης πράγμα που από μόνο του δείχνει καλύτερη επιλογή στόχων και ικανό-τητα του πράκτορα να μην αφήνει ακάλυπτα σημεία στο διάβα του. Τέλος ο μέσος χρόνος σε σχέση με την τυχαία μέθοδο κάλυψης έχει μειωθεί κατά πάνω από 25\% πράγμα που επιβεβαιώνει προηγούμενες μελέτες επί του θέματος.
Στη μέση επίδοση χρειαζόμαστε 15 λεπτά και 30 δευτερόλεπτα για την κάλυψη του χώρου (σε ποσοστό μεγαλύτερου του 95\%).

\newpage

\subsection{Πειράματα ενός πράκτορα - Μεγάλος Χάρτης}

Αρχικά και πάλι θα παρουσιάσουμε την τελική κατάσταση του χάρτη η οποία φαίνεται στην επόμενη εικόνα (Εικόνα 26). Το ροζ και πάλι προφανώς είναι το επίθεμα της κάλυψης.

\begin{figure}[!hb]
	\centering
	\includegraphics[scale=0.3]{assets/images/big-map-after-coverage.png}
	\caption{Τελική μορφή μεγάλου χάρτη μετά την ολοκλήρωση της κάλυψης}
\end{figure}

Εδώ παρατηρούμε ότι μένουν κάποια κενά εκεί στον μεγάλο χώρο στη μέση του χάρτη τουλάχιστον για το πρώτο 95\% της εξερεύνησης. Αυτό συμβαίνει διότι εκεί εξαιτίας της έλλειψης τοίχων δεν δούλεψε όπως έπρεπε το διάγραμμα Voronoi άρα και δεν προέκυψαν στόχοι σε σημεία εκεί μέσα. Παρ' όλα αυτά αν κανείς προσέθετε <<ψεύτικα>> εμπόδια πάνω στον χάρτη τότε θα προέκυπταν και εκεί στόχοι. Βέβαια αν δεν διακόψουμε την διεξαγωγή του πειράματος τελικά φυσικά καλύπτεται όλος ο χώρος με την μέθοδο της προσέγγισης του κοντινότερου ακάλυπτου σημείου. 

\subsubsection{Τυχαία επιλογή στόχων}

Πάλι ξεκινάμε με την τυχαία επιλογή και συνεχίζουμε με την έξυπνη μέθοδο

%\begin{figure}[!h]	
%	\includegraphics[scale=0.3]{assets/figures/rand-one-big-map-1.png}
%	\caption{κάλυψη ανά χρόνο μεγάλος χάρτης 1 πράκτορας τυχαία πείραμα 1}
%\end{figure}

\begin{figure}[!h]
	\centering
	\includegraphics[scale=0.3]{assets/figures/rand-one-big-map-2.png}
	\caption{Κάλυψη ανά χρόνο, μεγάλος χάρτης, 1 πράκτορας, τυχαία, ενδεικτικό πείραμα	}
\end{figure}

%\begin{figure}[!h]
%	\centering
%	\includegraphics[scale=0.3]{assets/figures/rand-one-big-map-3.png}
%	\caption{κάλυψη ανά χρόνο μεγάλος χάρτης 1 πράκτορας τυχαία πείραμα 3}
%\end{figure}

\begingroup
\centering
\begin{tabular}{c | c }
	\textbf{Λεκτική περιγραφή πειράματος} & \textbf{Χρόνοι}\\ \hline{}
	Καλύτερη επίδοση & 4049.171 sec \\ \hline
	Μέση επίδοση & 4472.695 sec   \\ \hline
	Χειρότερη επίδοση & 5052.156 sec \\ 
%	\caption{μεγάλος χάρτης 1 πράκτορας τυχαία σύνοψη}
\end{tabular}
\captionof{table}{Μεγάλος χάρτης, 1 πράκτορας, τυχαία, σύνοψη}
\endgroup

\paragraph{Σχολιασμός Πειραμάτων}

Εδώ εξαιτίας του μεγέθους του χάρτη ο χρόνος εκτέλεσης των πειραμάτων αυξάνει. Επίσης παρατηρούμε, πως από το 80\% μέχρι το τελικό 95\% που είναι και το ζητούμενο, ότι ο ρυθμός μεταβολής του ποσοστού κάλυψης αρχίζει να πέφτει. Αυτό συμβαίνει γιατί από εκεί και πέρα έπαψαν να δημιουργούνται κόμβοι οπότε και άρχισε να χρησιμοποιείται η μέθοδος κοντινότερου ακάλυπτου σημείου, και συνυπολογιζόμενης και της τυχαιότητας ο ρυθμός όντως πέφτει. Βάσει αυτής της παρατήρησης, εξάγουμε το συμπέρασμα ότι η μέθοδος του τοπολογικού γράφου είναι αποδοτικότερη μιας και δίνει την δυνατότητα να επιλέξουμε στόχους κλειδιά και βαθύτερα πέρα από τα όρια κάλυψης.


%%%%%%%%%%%%%%%%%%%%%%%%%%%%%%%%%%%%%%%%%%%%%%%%%%%%%%%%%%%%%%%%%%%%%%%%%%%%%%
%%%%%%%%%%%%%%%%%%%% Four Gains one rob BIG map %%%%%%%%%%%%%%%%%%%%%%%%%%%%
%%%%%%%%%%%%%%%%%%%%%%%%%%%%%%%%%%%%%%%%%%%%%%%%%%%%%%%%%%%%%%%%%%%%%%%%%%%%%%

\subsubsection{Επιλογή στόχων μέσω κοντινότερου κόμβου}

Εδώ παρουσιάζονται τα πειράματα όπου η επιλογή στόχων έγινε με την μέθοδο του κοντινό-τερου κόμβου. Θα δούμε ότι οι χρόνοι είναι σαφώς καλύτεροι. 


\begin{figure}[!h]	
	\includegraphics[scale=0.3]{assets/figures/four-one-big-map-1.png}
	\caption{Κάλυψη ανά χρόνο, μεγάλος χάρτης, 1 πράκτορας, κοντινότερος κόμβος, ενδεικτικό πείραμα}
\end{figure}

%\begin{figure}[!h]
%	\centering
%	\includegraphics[scale=0.3]{assets/figures/four-one-big-map-2.png}
%	\caption{κάλυψη ανά χρόνο μεγάλος χάρτης 1 πράκτορας κοντινότερος κόμβος πείραμα 2}
%\end{figure}

%\begin{figure}[!h]
%	\centering
%	\includegraphics[scale=0.3]{assets/figures/four-one-big-map-3.png}
%%	\caption{κάλυψη ανά χρόνο μεγάλος χάρτης 1 πράκτορας κοντινότερος κόμβος πείραμα 3}
%\end{figure}


\begingroup
\centering
\begin{tabular}{c | c }
	\textbf{Λεκτική περιγραφή πειράματος} & \textbf{Χρόνοι}\\ \hline{}
	Καλύτερη επίδοση & 2872.432 sec \\ \hline
	Μέση επίδοση & 3092.744 sec   \\ \hline
	Χειρότερη επίδοση & 3405.830 sec \\
%	\caption{μεγάλος χάρτης 1 πράκτορας κοντινότερος κόμβος σύνοψη} 
\end{tabular}
\captionof{table}{Μεγάλος χάρτης, 1 πράκτορας, κοντινότερος κόμβος, σύνοψη}
\endgroup


\paragraph{Σχολιασμός Πειραμάτων}

Εδώ παρατηρούμε ότι με την έξυπνη μέθοδο ο χρόνος μειώθηκε περαιτέρω και ο χάρτης καλύπτεται σε μέσο χρόνο κάτι λιγότερο από πενήντα δύο λεπτά.

\newpage

\subsection{Πειράματα δύο πρακτόρων - Μικρός Χάρτης}

Εδώ η τελική μορφή του χάρτη είναι ακριβώς η ίδια όποτε θα περάσουμε αμέσως στον σχολιασμό των πειραματικών δεδομένων. 
%Πρέπει να σημειωθεί ότι εδώ η μέτρηση του χρόνου έγινε κάπως διαφορετικά. Επειδή το σύστημα είναι κατανεμημένο τις χρονικές περιόδους που κινούνται και οι δύο πράκτορες έπρεπε να μετράται ο χρόνος μόνο από τον ένα. Τελικά όμως θεωρήθηκε καλή προσέγγιση ο μέσος όρος των 2 διαδρομών. Δηλαδή αν στην n διαδρομή το πρώτο ρομπότ έκανε χ δευτερόλεπτα και το δεύτερο ρομπότ έκανε y τότε ο χρόνος που καταχωρούνταν ήταν $ \frac{x + y}{2} $.

\subsubsection{Τυχαία επιλογή στόχων}

\begin{figure}[!h]	
	\includegraphics[scale=0.3]{assets/figures/rand-two-med-map-1.png}
	\caption{Κάλυψη ανά χρόνο, μικρός χάρτης, 2 πράκτορες, τυχαία, ενδεικτικό πείραμα }
\end{figure}

%\begin{figure}[!h]
%	\centering
%	\includegraphics[scale=0.3]{assets/figures/rand-two-med-map-2.png}
%	\caption{κάλυψη ανά χρόνο μικρός χάρτης 2 πράκτορες τυχαία πείραμα 2}
%\end{figure}

%\begin{figure}[!h]
%	\centering
%	\includegraphics[scale=0.3]{assets/figures/rand-two-med-map-3.png}
%	\caption{κάλυψη ανά χρόνο μικρός χάρτης 2 πράκτορες τυχαία πείραμα 3}
%\end{figure}


\begingroup
\centering
\begin{tabular}{c | c }
	\textbf{Λεκτική περιγραφή πειράματος} & \textbf{Χρόνοι}\\ \hline{}
	Καλύτερη επίδοση & 928.470 sec \\ \hline
	Μέση επίδοση & 985.291 sec   \\ \hline
	Χειρότερη επίδοση & 1077.812 sec \\
\end{tabular}
\captionof{table}{Μικρός χάρτης, 2 πράκτορες, τυχαία σύνοψη}
\endgroup
\paragraph{Σχολιασμός Πειραμάτων}

Εδώ παρατηρούμε ότι σε σχέση με την περίπτωση του ενός πράκτορα και της τυχαίας επιλογής στόχων προφανώς έχουμε καλύτερη επίδοση. Συγκριτικά μέσος χρόνος στην τυχαία επιλογή στόχων ενός πράκτορα στον μικρό χάρτη, ήταν είκοσι έξι λεπτά, ενώ εδώ διάρκεσε δέκα έξι και μισό λεπτά περίπου. Αξίζει φυσικά όμως να σημειωθεί ότι αλγόριθμος κοντινότερου κόμβου και στον ένα πράκτορα πήγε καλύτερα (κάτω από 16 λεπτά) απ' ότι οι δύο πράκτορες τυχαία! Αυτό αποδεικνύει ότι το πλεόνασμα σε υλικό (hardware) δεν λύνει τα προβλήματα από μόνο του. 

\newpage


%%%%%%%%%%%%%%%%%%%%%%%%%%%%%%%%%%%%%%%%%%%%%%%%%%%%%%%%%%%%%%%%%%%%%%%%%%%%%%
%%%%%%%%%%%%%%%%%%%% Four Gains TWO rob small map %%%%%%%%%%%%%%%%%%%%%%%%%%%%
%%%%%%%%%%%%%%%%%%%%%%%%%%%%%%%%%%%%%%%%%%%%%%%%%%%%%%%%%%%%%%%%%%%%%%%%%%%%%%

\subsubsection{Επιλογή στόχων μέσω κοντινότερου κόμβου}

Εδώ παρουσιάζονται τα πειράματα όπου η επιλογή στόχων έγινε με την μέθοδο του κοντινότερου κόμβου. Θα δούμε ότι οι χρόνοι είναι σαφώς καλύτεροι. 


\begin{figure}[!h]	
	\includegraphics[scale=0.3]{assets/figures/four-two-med-map-1.png}
	\caption{Κάλυψη ανά χρόνο, μικρός χάρτης, 2 πράκτορες, κοντινότερος κόμβος, ενδεικτικό πείραμα }
\end{figure}

%\begin{figure}[!h]
%	\centering
%	\includegraphics[scale=0.3]{assets/figures/four-two-med-map-2.png}
%	\caption{κάλυψη ανά χρόνο μικρός χάρτης 2 πράκτορες κοντινότερος κόμβος πείραμα 2}
%\end{figure}

%\begin{figure}[!h]
%	\centering
%	\includegraphics[scale=0.3]{assets/figures/four-two-med-map-3.png}
%	\caption{κάλυψη ανά χρόνο μικρός χάρτης 2 πράκτορες κοντινότερος κόμβος πείραμα 3}
%\end{figure}

\begingroup
\centering
\begin{tabular}{c | c }
	\textbf{Λεκτική περιγραφή πειράματος} & \textbf{Χρόνοι}\\ \hline{}
	Καλύτερη επίδοση & 694.358 sec \\ \hline
	Μέση επίδοση & 724.156 sec   \\ \hline
	Χειρότερη επίδοση & 772.507 sec \\ 
\end{tabular}
\captionof{table}{Μικρός χάρτης, 2 πράκτορες, κοντινότερος κόμβος, σύνοψη}
\endgroup

\paragraph{Σχολιασμός Πειραμάτων}

Εδώ δεν βλέπουμε την βελτίωση που περιμέναμε με την εισαγω-γή του δεύτερου πράκτορα σε σχέση με τα αποτελέσματα της μεθόδου με τον έναν πράκτορα. Αυτό συμβαίνει γιατί εξαιτίας περιορισμένης επεξεργαστικής δύναμης ενώ οι πράκτορες περνούσαν από μονοπάτι δεν το θεωρούσε αμέσως ο αλγόριθμος καλυμμένο, και κατά δεύτερον τα δύσκολα σημεία που πρέπει να καλύψουν οι πράκτορες παραμένουν δυσπρόσιτα σημεία οπότε η εισαγωγή του πράκτορα δεν συμβάλει και πολλά σε αυτό. Σίγουρα όμως οι μετρήσεις είναι ενθαρρυντικές.

\newpage

\subsection{Πειράματα δύο πρακτόρων - Μεγάλος Χάρτης}

Και πάλι θα φανεί ότι η προσθήκη του δεύτερου πράκτορα δεν μειώνει τον χρόνο όσο θα περίμενε κανείς.

\subsubsection{Τυχαία επιλογή στόχων}

%\begin{figure}[!h]	
%	\includegraphics[scale=0.3]{assets/figures/rand-two-big-map-1.png}
%	\caption{κάλυψη ανά χρόνο μεγάλος χάρτης 2 πράκτορες τυχαία πείραμα 1}
%\end{figure}

\begin{figure}[!h]
	\centering
	\includegraphics[scale=0.3]{assets/figures/rand-two-big-map-2.png}
	\caption{Κάλυψη ανά χρόνο, μεγάλος χάρτης, 2 πράκτορες, τυχαία, ενδεικτικό πείραμα}
\end{figure}

%\begin{figure}[!h]
%	\centering
%	\includegraphics[scale=0.3]{assets/figures/rand-two-big-map-3.png}
%	\caption{κάλυψη ανά χρόνο μεγάλος χάρτης 2 πράκτορες τυχαία πείραμα 3}
%\end{figure}


\begingroup
\centering
\begin{tabular}{c | c }
	\textbf{Λεκτική περιγραφή πειράματος} & \textbf{Χρόνοι}\\ \hline{}
	Καλύτερη επίδοση & 2896.984 sec \\ \hline
	Μέση επίδοση & 2904.805 sec   \\ \hline
	Χειρότερη επίδοση & 3021.140 sec \\
\end{tabular}
\captionof{table}{Μεγάλος χάρτης, 2 πράκτορες, τυχαία, σύνοψη}
\endgroup
\paragraph{Σχολιασμός Πειραμάτων}

Και πάλι παρατηρούμε σε σχέση με την τυχαία επιλογή ενός πράκτορα (Μέσος χρόνος 4472.695 sec) βλέπουμε σαφώς βελτίωση.  

\newpage


%%%%%%%%%%%%%%%%%%%%%%%%%%%%%%%%%%%%%%%%%%%%%%%%%%%%%%%%%%%%%%%%%%%%%%%%%%%%%%
%%%%%%%%%%%%%%%%%%%% Four Gains TWO rob BIG map %%%%%%%%%%%%%%%%%%%%%%%%%%%%
%%%%%%%%%%%%%%%%%%%%%%%%%%%%%%%%%%%%%%%%%%%%%%%%%%%%%%%%%%%%%%%%%%%%%%%%%%%%%%

\subsubsection{Επιλογή στόχων μέσω κοντινότερου κόμβου}

%\begin{figure}[!h]	
%	\includegraphics[scale=0.3]{assets/figures/four-two-big-map-1.png}
%%	\caption{κάλυψη ανά χρόνο μεγάλος χάρτης 2 πράκτορες κοντινότερος κόμβος πείραμα 1}
%\end{figure}

%\begin{figure}[!h]
%	\centering
%	\includegraphics[scale=0.3]{assets/figures/four-two-big-map-2.png}
%	\caption{κάλυψη ανά χρόνο μεγάλος χάρτης 2 πράκτορες κοντινότερος κόμβος πείραμα 2}
%\end{figure}

\begin{figure}[!h]
	\centering
	\includegraphics[scale=0.3]{assets/figures/four-two-big-map-3.png}
	\caption{Κάλυψη ανά χρόνο, μεγάλος χάρτης, 2 πράκτορες κοντινότερος, κόμβος, ενδεικτικό πείραμα}
\end{figure}

\begingroup
\centering
\begin{tabular}{c | c }
	\textbf{Λεκτική περιγραφή πειράματος} & \textbf{Χρόνοι}\\ \hline{}
	Καλύτερη επίδοση & 2221.155 sec \\ \hline
	Μέση επίδοση & 2417.332 sec   \\ \hline
	Χειρότερη επίδοση & 2765.804 sec \\ 
\end{tabular}
\captionof{table}{Μεγάλος χάρτης, 2 πράκτορες, κοντινότερος κόμβος, σύνοψη}
\endgroup

\paragraph{Σχολιασμός Πειραμάτων}

Πάλι εδώ παρατηρούμε αρκετά καλή βελτίωση σε σχέση με την ίδια μέθοδο επιλογής στόχων με ένα πράκτορα. Αυτό συμβαίνει γιατί εφόσον είναι μεγάλος ο χάρτης οι χρονικές διαφορές φαίνονται ευκολότερα. Τέλος φαίνεται ότι η χρήση της προσέγγισης κοντινότερου ακάλυπτου σημείου στο τέλος της προσέγγισης δεν φαίνεται να επηρεάζει τόσο αρνητικά την κατάσταση. 


\subsection{Συνολική παρουσίαση αποτελεσμάτων}

\begingroup
\centering
\begin{tabular}{c | c | c }
	\textbf{1 Πράκτορας μικρός χάρτης} & \textbf{Τυχαία επιλογή} & \textbf{Κοντινότερος κόμβος} \\ \hline
	Καλύτερη επίδοση & 1365.899 sec & 790.553 sec \\ \hline
	Μέση επίδοση & 1561.642 sec  & \textbf{931.56 sec}   \\ \hline
	Χειρότερη επίδοση & 1693.71 sec & 1086.215 sec \\
	\textbf{1 Πράκτορας μεγάλος χάρτης} & \textbf{Τυχαία επιλογή} & \textbf{Κοντινότερος κόμβος}\\ \hline
	Καλύτερη επίδοση & 4049.171 sec & 2872.432 sec \\ \hline
	Μέση επίδοση & 4472.695 sec   & \textbf{3092.744 sec}   \\ \hline
	Χειρότερη επίδοση & 5052.156 sec & 3405.830 sec \\
	\textbf{2 Πράκτορες μικρός χάρτης} & \textbf{Τυχαία επιλογή} & \textbf{Κοντινότερος κόμβος} \\ \hline
	Καλύτερη επίδοση & 928.470 sec & 694.358 sec \\ \hline
	Μέση επίδοση & 985.291 sec & \textbf{ 724.156 sec}   \\ \hline
	Χειρότερη επίδοση & 1077.812 sec & 772.507 sec \\ 
	\textbf{2 Πράκτορες μεγάλος χάρτης} & \textbf{Τυχαία επιλογή} & \textbf{Κοντινότερος κόμβος}\\ \hline
	Καλύτερη επίδοση & 2896.984 sec & 2221.155 sec \\ \hline
	Μέση επίδοση & 2904.805 sec  & \textbf{2417.332 sec}   \\ \hline
	Χειρότερη επίδοση & 3021.140 sec & 2765.804 sec \\ 
	
	
	
\end{tabular}
\captionof{table}{Συνοπτικά αποτελέσματα}
\endgroup

\begin{figure}[!h]
	\centering
	\includegraphics[scale=0.3]{assets/figures/one_rob_med_map.png}
	\caption{Σύγκριση μεθόδων, μικρός χάρτης, 1 ρομπότ}
\end{figure}

\newpage

\begin{figure}[!h]
	\centering
	\includegraphics[scale=0.3]{assets/figures/one_rob_big_map.png}
	\caption{Σύγκριση μεθόδων, μεγάλος χάρτης, 1 ρομπότ}
\end{figure}

\begin{figure}[!h]
	\centering
	\includegraphics[scale=0.3]{assets/figures/two_rob_med_map.png}
	\caption{Σύγκριση μεθόδων, μικρός χάρτης, 2 ρομπότ}
\end{figure}

\newpage

\begin{figure}[!h]
	\centering
	\includegraphics[scale=0.3]{assets/figures/two_rob_big_map.png}
	\caption{Σύγκριση μεθόδων, μεγάλος χάρτης, 2 ρομπότ}
\end{figure}

\paragraph{Σχολιασμός Τελικών Αποτελεσμάτων}

Παρατηρούμε ότι σε κάθε περίπτωση η προσέγ-γιση κοντινότερου κόμβου, δίνει καλύτερα αποτελέσματα ακριβώς γιατί έχει και μια ευφυΐα. Κάποιες φορές παρατηρούμε ότι στην αρχή της εξερεύνησης πάει καλύτερα η τυχαία μέθοδος όμως αυτό συμβαίνει γιατί καθώς διαλέγει τυχαία στόχους, διανύει πιο μεγάλες αποστάσεις και κερδίζει κάλυψη πιο γρήγορα. Στο τέλος όμως έχει να γυρίσει και να καλύψει ακάλυπτα σημεία όποτε εκεί υστερεί και κερδίζει η μέθοδος κοντινότερου κόμβου.
Στο επόμενο κεφάλαιο θα δούμε τα προβλήματα που αντιμετωπίστηκαν και ευκαιρίες για μελλοντική μελέτη.
