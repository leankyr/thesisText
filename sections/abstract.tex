\section*{Abstract}

\begin{minipage}{1\textwidth}
\centering
\LARGE{\textbf{Full Coverage of an a-priori Known Map From Multiple Robotic Agents}}
\end{minipage}

\vspace{1 cm}

Robotics is a field that gets a lot of attention in the last decades. One of the first problems that concerned researchers were robotic coverage and localization. Those problems are equally interesting today and in addition to the trivial single agent implementations, we get to see multi-agent implementations too. Those solutions are applied to different needs, for instance, military applications, urban search and rescue (USAR), perimeter surveillance. Those efforts got even more plausible in the last decade due to the increase of processing power, the fall of hardware prices and the increase of network bandwidth.

The current thesis handles the problem of coordinating two agents with the goal of fully covering an a priori known space with the use of a sensor of interest. In the first chapter we present the introduction, the purpose of this thesis and the layout of the text. In the second chapter we present the state of the art concerning robotic exploration, task allocation and robotic coverage. In the third we describe the tools used specifically the hardware, ROS and its libraries. In the fourth chapter we present the implementation used - the topological graph and the closest frontier-based coverage.  In the fifth chapter results are presented and discussed. Finally we comment on the problems faced, possibilities for future work and the conclusions drawn.

For the simulation part Gazebo 3D simulator and ROS's rviz were utilized. For the coding part Python scripting language was deployed. 

\begin{flushright}
Georgios Leandros Kyriazis \\
Aristotle University of Thessaloniki \\
Dept. of Electrical Engineering \& Computer Engineering \\
leankyr@gmail.com \\
June 2019	
\end{flushright} 
 