\section{Προβλήματα \& Μελλοντική Εργασία}

\subsection{Προβλήματα}

\paragraph{}Προβλήματα ποικίλης φύσεως προέκυψαν κατά την εκπόνηση της παρούσας εργασίας. 

Συγκριτικά αρκετός χρόνος δαπανήθηκε στην προσαρμογή των παραμέτρων, στην σύν-δεση μεταξύ των διάφορων πακέτων, και στην εύρεση κατάλληλου σχεδιαστή διαδρομής (path planner). Για παράδειγμα για την εξαγωγή του γνωστού χάρτη χρειαζόταν αρκετά καλή ανάλυση για να δουλέψει έπειτα ο τοπολογικός γράφος. Η καλή ανάλυση όμως απαιτούσε και επιμονή κατά την διαδικασία της εξερεύνησης. Επίσης είχε γίνει προσπάθεια να χρησιμοποιηθούν πακέτα πηγαίου κώδικα αλλά τελικά αποδείχτηκε καλύτερη η χρήση των πακέτων που διατίθεντο μέσω του λογισμικού Ubuntu. Τέλος πειραματιστήκαμε με πολλούς τοπικούς σχεδιαστές διαδρομών (local path planner), και καταλήξαμε στον TEB μετά από τον DWA και τον basic local planner.

Επίσης κάποιος θα μπορούσε να αναρωτηθεί εφόσον ο χάρτης είναι γνωστός γιατί να μην είναι οι στόχοι προυπολογισμένοι από πριν και απλά οι πράκτορες να του περιηγούνται με την σειρά. Η απάντηση είναι ότι ευκταίο είναι το παρόν έργο να χρησιμοποιηθεί τελικά σε άγνωστο περιβάλλον οπού αυτό το ενδεχόμενο θα είναι ανέφικτο. Εκεί όμως αυτή η μέθοδος επιλογής στόχων θα δουλεύει όπως δούλεψε και στον γνωστό χάρτη.

Τέλος η έλλειψη επεξεργαστικής δύναμης κατέστησε την εκτέλεση των πειραμάτων ελαφρώς χρονοβόρα.


\subsection{Μελλοντική Εργασία}

\paragraph{} Πολλές θα μπορούσαν να είναι οι προοπτικές εξέλιξης της παρούσας εργασίας. Μια προφανής θα ήταν η προσθήκη περισσότερων αισθητήρων πάνω στο πράκτορα για ακόμα ποιοτικότερη κάλυψη αλλά και ανίχνευση διαφορετικής πληροφορίας. 
Στο ήδη υπάρχον σύστημα ακόμη θα μπορούσε να ενταχθεί ένας κόμβος υπολογιστικής όρασης με σκοπό οι πράκτορες να μπορούν να ανιχνεύουν διάφορα αντικείμενα. Επίσης θα μπορούσε να στερεωθεί και ένας ρομποτικός βραχίωνας για ανάπτυξη εφαρμογών σήκωσε-και-τοποθέτη-σε (pick-and-place). 

Όσο αναφορά την κάλυψη η ίδια ιδέα θα μπορούσε να εφαρμοστεί και σε προβλήματα εξερεύνησης σε SLAM. Απλά εκεί αντί να ενώνονται τα πεδία κάλυψης θα ενώνονται οι χάρτες. Φυσικά αυτό είναι ένα δυσκολότερο πρόβλημα που πρέπει να ληφθεί υπόψιν η θέση των πρακτόρων στον σταδιακά εξερευνημόμενο χώρο, καθώς επίσης και τον προσανατολι-σμό των διάφορων χαρτών. Παρόλα αυτά ήδη γνωρίζουμε ότι υπάρχουν κάποιες λύσεις. Επίσης χρήσιμο θα ήταν η εφαρμογή να δοκιμαστεί και σε πραγματικές συνθήκες και να ελεγχθεί και εκεί η αποδοτικότητά της.

Τέλος ενδιαφέρουσα θα ήταν η δοκιμή εκτέλεσης των αλγορίθμων σε ένα μια cloud υποδομή για προσθήκη μεγαλύτερης  επεξεργαστικής δύναμης αλλά και για ευκολότερο συντονισμό ακόμα περισσότερων πρακτόρων.

Στο επόμενο κεφάλαιο παρουσιάζονται τα συμπεράσματα που προέκυψαν.

 