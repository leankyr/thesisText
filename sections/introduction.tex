\section{Εισαγωγή}

\paragraph{}Η Μηχανική Ευφυΐα ήταν ένα ζήτημα το οποίο συναρπάζει το μυαλό του ανθρώπου από παλαιοτάτων χρόνων. Οι αρχαίοι Έλληνες μιλούσαν για τα κινούμενα αγάλματα του Δαίδαλου και τους τρίποδες του Ήφαιστου οι οποίοι βρίσκονταν στον Όλυμπο και υπηρετού-σαν τους Θεούς. Έκτοτε διάφοροι άνθρωποι, λόγου χάριν  οι Asimov, Clarke και Le Guin, έφτιαξαν ιστορίες για το πως θα ήταν μελλοντικά οι μηχανές. Σήμερα, με την ανάπτυξη της επιστήμης των υπολογιστών και συγκεκριμένα με την επιστήμη της Ρομποτικής αυτές οι ιστορίες τείνουν να πραγματοποιηθούν. 

Η Ρομποτική εδώ και αρκετά χρόνια χρησιμοποιείται στην Βιομηχανία, παραδείγματος χάριν στις γραμμές παραγωγής αυτοκινήτων ή ηλεκτρικών συσκευών, αλλά μόλις τα τελευ-ταία 30 χρόνια βλέπουμε προσπάθειες για εφαρμογή ευφυών πρακτόρων (π.χ. σκούπα roomba), δηλαδή πρακτόρων οι οποίοι μπορούν να λάβουν αποφάσεις μόνοι τους χωρίς την ανθρώπινη επίβλεψη. Οι πράκτορες στις γραμμές παραγωγής λέγονται μή ευφυείς μιας και αδυνατούν να κατανοήσουν το περιβάλλον γύρω τους. Οι ευφυείς πράκτορες προκείμενου να αντιλαμβάνονται τον χώρο γύρω τους εφοδιάζονται με διάφορους αισθητήρες για την πλοήγηση, συνηθέστερα Lidar και σπανιότερα άλλους. Μάλιστα κάποιες φορές η πλοήγηση τους γίνεται με τρόπο αρκετά απλοϊκό. Τέλος ανάλογα με το έργο που καλούνται να επιτελέ-σουν, οι υπόλοιποι αισθητήρες μπορούν να ποικίλουν. 

Αν και έχουν γίνει αρκετές προσπάθειες για μεμονωμένους πράκτορες, εξίσου πολλές έχουν γίνει και για εφαρμογές πολλαπλών πρακτόρων, σύμφωνα με όσα γνωρίζει ο συγ-γραφέας. Είναι προφανές ότι με μεγαλύτερο αριθμό αυτόνομων πρακτόρων, μπορούν να λυθούν τα ίδια προβλήματα γρηγορότερα αλλά απαιτείται συντονισμός αυτών και διάφορες βελτιστοποιήσεις σχετικά με την πλοήγηση τους στον εκάστοτε χώρο.

Η παρούσα διπλωματική εργασία ασχολείται με συντονισμό πολλαπλών πρακτόρων και βέλτιστη επιλογή στόχων.  

\subsection{Σκοπός της εργασίας}

\paragraph{}Σκοπός της διπλωματικής αυτής εργασίας είναι η κάλυψη ενός γνωστού εκ των προτέρων χώρου - του οποίου διαθέτουμε τον χάρτη - μέσω ενός αισθητήρα ενδιαφέροντος και ενός Lidar που χρησιμοποιείται στην πλοήγηση. Αρχικά με το Lidar και μέσω του αλγορίθμου του amcl, το ρομπότ βρίσκει την θέση του μέσα στο χάρτη. Ταυτόχρονα καθώς κινείται ο πράκτορας, ο αισθητήρας καλύπτει τον χώρο και η πληροφορία αυτή αποθηκεύεται στην μνήμη.

Έπειτα, ανάλογα με τον καλυμμένο χώρο και την θέση των εμποδίων στο χάρτη, υπολογί-ζεται ο τοπολογικός γράφος που θα περιγραφεί σε επόμενο κεφάλαιο, με σκοπό οι πράκτορες να επιλέξουν στόχο. Υπάρχει συνεχής επικοινωνία μεταξύ των πρακτόρων ώστε να μην προσεγγίζουν κοντινούς στόχους. Ταυτόχρονα το ίδιο Lidar χρησιμοποιείται για την αποφυ-γή εμποδίων.  
 
\subsection{Συνεισφορά της εργασίας}

Η εργασία αυτή σίγουρα θα μπορούσε να βοηθήσει στην επιτάχυνση επίλυσης προβλημάτων που ήδη λύνονται από μεμονωμένους πράκτορες. Και επίσης η χρήση πολλαπλών πρακτό-ρων προσφέρει μεγάλη ευελιξία στην χρήση. Ένα παράδειγμα θα ήταν η χρήση σε καθαρισμό πολύ μεγάλων χώρων π.χ. αεροδρομίων ή σταθμών τρένων.

Επίσης η ύπαρξη κάμερας προσφέρει δυνατότητες αναγνώρισης εικόνας. Η αναγνώριση εικόνας ίσως βοηθούσε σε εκθεσιακούς και μουσειακούς χώρους οπού δεν δίνεται η δυνατό-τητα φωτογράφισης των εκθεμάτων από τους πελάτες, οι πράκτορες με κατάλληλα ρυθμισμ-ένες φωτογραφικές μηχανές, θα μπορούσαν να φωτογραφίζουν τα εκθέματα για αυτούς.

Με χρήση άλλου αισθητήρα, π.χ. θερμικής κάμερας, θα μπορούσε να γίνει συλλογή δεδομένων σε έναν εξωτερικό χώρο. Τέλος αν υπάρχει δυνατότητα συγχώνευσης χαρτών η εφαρμογή θα μπορούσε να χρησιμοποιηθεί και σε αγνώστους χώρους.


\subsection{Διάρθρωση της εργασίας}

Η εργασία περιλαμβάνει εφτά κεφάλαια συμπεριλαμβανομένου αυτού εδώ τα οποία διαρθρώ-νονται ως εξής:

\begin{itemize}
	\item \textbf{Κεφάλαιο 2:} επισκόπηση της ερευνητικής περιοχής 
	\item \textbf{Κεφάλαιο 3:} παρουσίαση εργαλείων που χρησιμοποιήθηκαν
	\item \textbf{Κεφάλαιο 4:} Ανάλυση των αλγορίθμων που εφαρμόστηκαν για την επίλυση του προβλήματος 
	\item \textbf{Κεφάλαιο 5:} Παρουσίαση πειραμάτων και αποτελέσματα 	
	\item \textbf{Κεφάλαιο 6:} αναφορά στα προβλήματα που παρουσιάστηκαν και προοπτικές για μελλοντική μελέτη
	\item \textbf{Κεφάλαιο 7:} Καταγραφή των συμπερασμάτων αυτής της μελέτης
\end{itemize}

\newpage
